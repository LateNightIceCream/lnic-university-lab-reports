\documentclass[11pt, a4paper]{article}

%
%   PACKAGES
%

\usepackage{graphicx}
\usepackage{amsmath}
\usepackage{amssymb}
\usepackage{mathtools}
\usepackage[binary-units=true]{siunitx}
\usepackage[table,xcdraw]{xcolor}
\usepackage{tikz}
\usepackage[most]{tcolorbox}
%\usepackage[english,german]{babel}
\usepackage[english]{babel}
\usepackage{blindtext}
\usepackage{import}
\usepackage{float}
\usepackage{varwidth}
\usepackage{esint}
\usepackage{subfig}
\usepackage[hyphens]{url}
\usepackage{hyperref}
\usepackage{pdfpages}
\usepackage{multicol}
\usepackage{listings}
\usepackage{geometry}
\usepackage{fontspec}
\usepackage[
    backend=biber,
]{biblatex}
\usepackage{todonotes}

%
%   GEOMETRY
%

\geometry{
  a4paper,
  %total={0.618\paperwidth,0.7639\paperheight},
  %left=0.190962\paperwidth,
  total={0.764\paperwidth,0.7639\paperheight},
  left=0.118\paperwidth,
}


%
%   FONT
%

%s\renewcommand{\familydefault}{\sfdefault}

%
%   COLORS
%

% \input{"/home/zamza/Documents/HS/Master/lnic-masters-protocols/preamble/colors.tex"}
\input{"colors.tex"}
\definecolor{yucky}{HTML}{dee2e6}
\definecolor{yuckytext}{HTML}{343a40}

%
%   HIGHLIGHTING
%
\newcommand{\minisec}[1]{\noindent\underline{\textbf{#1}}\\}

%
%   MATH
%

\newcommand\equalhat{\mathrel{\stackon[1.5pt]{=}{\stretchto{%
    \scalerel*[\widthof{=}]{\wedge}{\rule{1ex}{3ex}}}{0.5ex}}}}

\newcommand{\vecnabla}{\vec{\nabla}}
\newcommand{\rot}{\text{rot}\,}
\newcommand{\divv}{\text{div}\,}
\newcommand{\grad}{\text{grad}\,}
\newcommand{\divD}{\divv\vec{D}}
\newcommand{\divB}{\divv\vec{B}}
\newcommand{\divE}{\divv\vec{E}}
\newcommand{\rotD}{\rot\vec{D}}
\newcommand{\rotB}{\rot\vec{B}}
\newcommand{\rotE}{\rot\vec{E}}
\newcommand{\unitv}{\vec{e}}
\newcommand{\partialdev}[2]{\frac{\partial #1}{\partial #2}}

% ! equals
\newcommand{\hastobe}{\stackrel{!}{=}}

% dBm


%
%   HYPERLINKS
%
\hypersetup{
    colorlinks=true,
    linkcolor=blue-6,
    citecolor=blue-8,
    urlcolor=blue-6
}


%
%   FLOWCHART
%

\usetikzlibrary{shapes,arrows}

\tikzstyle{decision} = [diamond, draw, fill=blue!20,
    text width=4.5em, text badly centered, node distance=3cm, inner sep=0pt]
\tikzstyle{block} = [rectangle, draw=yucky, fill=yucky, text=yuckytext,
    text width=10em, text centered, rounded corners, minimum height=4em, minimum width=10em]
\tikzstyle{line} = [draw, -latex']
\tikzstyle{cloud} = [draw, ellipse,fill=red!20, node distance=3cm,
    minimum height=2em]

%
%   NOTE BOX
%

\colorlet{tcb_content_bg}{green-0}
\colorlet{tcb_title_bg}{green-0}
%\tcbset{colback=tcb_content_bg, colbacktitle=tcb_title_bg, colframe=tcb_content_bg, boxrule=0pt, bottomrule=0pt, frame hidden, sharp corners}

\tcbset{
  enhanced,
  sharp corners,
}

\newcommand{\notebox}[2]{
  \vspace{\baselineskip}
  \begin{tcolorbox}[title=#1]{#2}\end{tcolorbox}
  \vspace{\baselineskip}
}

\colorlet{zitat_bg}{gray-0}
\colorlet{zitat_strip}{gray-3}

\newtcolorbox{taskspec}[1][]{%
    colback=gray-1,
    fontupper=\selectfont\ttfamily,
    %grow to right by=-10mm,
    %grow to left by=-10mm,
    boxrule=0pt,
    boxsep=0pt,
    breakable,
    enhanced jigsaw,
    borderline west={1pt}{0pt}{gray-2},
    borderline east={1pt}{0pt}{gray-2},
    borderline north={1pt}{0pt}{gray-2},
    borderline south={1pt}{0pt}{gray-2},
    %colbacktitle={gray-2},
    %coltitle={gray-8},
    %fonttitle={\large\bfseries},
    attach title to upper={},
    #1,
}

\newtcolorbox{zitat}[2][]{%
    colback=zitat_bg,
    grow to right by=-10mm,
    grow to left by=-10mm,
    boxrule=0pt,
    boxsep=0pt,
    breakable,
    enhanced jigsaw,
    borderline west={4pt}{0pt}{zitat_strip},
    title={#2\par},
    colbacktitle={zitat_bg},
    coltitle={gray-8},
    fonttitle={\large\bfseries},
    attach title to upper={},
    #1,
}

\newcommand{\notebo}[2]{
  \vspace{\baselineskip}
  \begin{tcolorbox}[enhanced,
  sharp corners,
  boxrule=0pt,
  toptitle=0.1cm+1pt,%
  bottomtitle=-0.1cm+0.5em,%
  colframe=red-0,colback=red-0,coltitle=red-7,
  title style=red-0,
  fonttitle=\bfseries,fontupper=\normalsize,title=#1]{#2}\end{tcolorbox}
  \vspace{\baselineskip}
}

\newtcbtheorem[number within=chapter]{thm}{Theorem}{
  theorem style=change apart,
  enhanced,
  frame hidden,interior hidden,
  sharp corners,
  boxrule=0pt,
  left=0.2cm,right=0.2cm,top=0.2cm,
  toptitle=0.1cm+1pt,%        <-- I used your values here
  bottomtitle=-0.1cm+0.5em,%  <-- I used your values here
  colframe=white!25!black,colback=white,coltitle=white,
  title style=white!25!black,
  bottomrule=1pt,%  <-- reserve space
  borderline south={1pt}{0pt}{white!25!black},%---- draw line
  fonttitle=\bfseries,fontupper=\normalsize}{thm}


%
%   CODE
%

\definecolor{commentColor}{HTML}{adb5bd}
\definecolor{mygray}{rgb}{0.5,0.5,0.5}
\definecolor{stringColor}{HTML}{7048e8}
\definecolor{keywordColor}{HTML}{228be6}
\definecolor{backgroundColor}{HTML}{f1f3f5}
\definecolor{borderColor}{HTML}{f1f3f5}
\definecolor{inlineTextColor}{HTML}{495057}
\definecolor{leftRuleColor}{HTML}{868e96}
\definecolor{numbackgroundColor}{HTML}{f1f3f5}
\definecolor{numColor}{HTML}{adb5bd}

\newtcbox{\inlinebox}{enhanced,nobeforeafter,tcbox raise base,boxrule=0pt,top=0.062em,bottom=0.062em,
  right=0.382em,left=0.382em,arc=0.382em,boxsep=0.1em,before upper={\vphantom{dlg}},
  colframe=white,colback=backgroundColor}

\newcommand{\inlinecode}[1] {
  \inlinebox{\lstinline[language=Python, identifierstyle=\color{inlineTextColor}, basicstyle=\color{inlineTextColor}\ttfamily, keywordstyle=\color{inlineTextColor}]{#1}}
}

%\setmonofont{JetBrainsMono NF}[
%    Contextuals = Alternate,
%    Ligatures = TeX,
%]

\lstset{
  backgroundcolor=\color{backgroundColor},   % choose the background color; you must add \usepackage{color} or \usepackage{xcolor}; should come as last argument
  basicstyle=\small\ttfamily,        % the size of the fonts that are used for the code
  breaklines=true,                 % sets automatic line breaking
  commentstyle=\color{commentColor},    % comment style
  extendedchars=true,              % lets you use non-ASCII characters; for 8-bits encodings only, does not work with UTF-8
  firstnumber=1,                % start line enumeration with line 1000
  frame=single,	                   % adds a frame around the code
  frameshape={RYR}{Y}{Y}{RYR},
  keepspaces=true,                 % keeps spaces in text, useful for keeping indentation of code (possibly needs columns=flexible)
  keywordstyle=\color{keywordColor}\textbf,       % keyword style
  language=Python,                 % the language of the code
  morekeywords={*,...},            % if you want to add more keywords to the set
  numbers=left,                    % where to put the line-numbers; possible values are (none, left, right)
  numbersep=1.5em,                   % how far the line-numbers are from the code
  numberstyle=\tiny\color{commentColor}, % the style that is used for the line-numbers
  rulecolor=\color{borderColor},         % if not set, the frame-color may be changed on line-breaks within not-black text (e.g. comments (green here))
  showstringspaces=false,          % underline spaces within strings only
  showtabs=false,                  % show tabs within strings adding particular underscores
  stepnumber=1,                    % the step between two line-numbers. If it's 1, each line will be numbered
  stringstyle=\color{stringColor},     % string literal style
  tabsize=2,
  frame=l,
  framesep=2.12em,
  framexleftmargin=0em,
  fillcolor=\color{numbackgroundColor},
  rulecolor=\color{leftRuleColor},
  numberstyle=\ttfamily\tiny\color{numColor},
}

\newtcolorbox{codebg}[2][]{%
    colback=gray-1,
    %grow to right by=-10mm,
    %grow to left by=-10mm,
    boxrule=0pt,
    boxsep=0pt,
    breakable,
    enhanced jigsaw,
    rounded corners=east,
    arc=8pt,
    borderline west={1pt}{0pt}{gray-3},
    %title={#2\par},
    %colbacktitle={zitat_bg},
    bottomrule=0pt,
}


\newtcbox{\inlinecodee}{on line, boxrule=0pt, boxsep=0pt, top=2pt, left=2pt, bottom=2pt, right=2pt, colback=gray-2, colframe=white, fontupper={\ttfamily \footnotesize}}

\BeforeBeginEnvironment{minted}{\begin{codebg}}%
\AfterEndEnvironment{minted}{\end{codebg}}%

\BeforeBeginEnvironment{inputminted}{\begin{codebg}}%
\AfterEndEnvironment{inputminted}{\end{codebg}}%

\usepackage{minted}
\setminted{
  autogobble=true,
  breakautoindent=true,
  breaklines=true,
  escapeinside=§§,
  fontfamily=tt,
  fontsize=\footnotesize,
  frame=leftline,
  framerule=0pt,
  framesep=0.2em, % sufficient for up to 4 digits
  numbers=left,
  numbersep=0.2em,
  showspaces=false,
  showtabs=false,
  style=vs, % see: https://pygments.org/styles/
  tabsize=2,
  xleftmargin=1.5em,
  % colors
  bgcolor=gray-1,
}
\usemintedstyle{myown}

\tcbuselibrary{minted}

% minted line numbers
\renewcommand{\theFancyVerbLine}{\sffamily
\textcolor{gray-4}{\scriptsize
\oldstylenums{\arabic{FancyVerbLine}}}}


\begin{document}

\includepdf{./titlepage/titlepage.pdf}

\section{Teil A: 2-Phasen-Scheduler}

\subsection{Prototyp 2-Phasen-Scheduler sched2p}

\lstinputlisting[title={sched2p.py}]{code/2phase.py}%[language=Python]

\begin{lstlisting}[title={Ausgabe sched2p bei Simulationszeit von 10 Minuten}]
time,state
0.0,TrafficLightStates.RED
180.12345,TrafficLightStates.GREEN
240.69134,TrafficLightStates.RED
420.81479,TrafficLightStates.GREEN
481.38268,TrafficLightStates.RED
\end{lstlisting}

\subsection{Zeitgetriebener Simulator (sched2p\_timedriven)}

\lstinputlisting[title={sched2p\_timedriven.py}]{code/2phase_timedriven.py}%[language=Python]

\begin{lstlisting}[title={Teil der Ausgabe sched2p\_timedriven bei Simulationszeit von 10 Minuten. Schrittweite = $1\,\si{\second}$}]
time,state
...
422.0,TrafficLightStates.RED
423.0,TrafficLightStates.RED
424.0,TrafficLightStates.RED
425.0,TrafficLightStates.GREEN
426.0,TrafficLightStates.GREEN
427.0,TrafficLightStates.GREEN
...
483.0,TrafficLightStates.GREEN
484.0,TrafficLightStates.GREEN
485.0,TrafficLightStates.GREEN
486.0,TrafficLightStates.RED
487.0,TrafficLightStates.RED
488.0,TrafficLightStates.RED
...
599.0,TrafficLightStates.RED
600.0,TrafficLightStates.RED
601.0,TrafficLightStates.RED
terminated
\end{lstlisting}

Die Ausgabe bei $601$ Sekunden liegt daran, dass im Simulationsschritt die Zeit nach dem Ausführen des Ereignisses erhöht wird. Die eigentliche Zeit bei Eintreten des Ereignisses liegt also immer $1\,\si{s}$ früher als es die Ausgabe zeigt.


\subsection{Vergleich von sched2p und sched2p\_timedriven}

Man erkennt gut aus der Ausgabe von \inlinecode{sched2p_timedriven}, dass die Zeitpunkte der Ereigniswechsel nicht genau getroffen werden können und die Genauigkeit immer von der Zeitauflösung / Schrittweite abhängt. Höhere Genauigkeit bringt allerdings auch eine erhöhte Anzahl an Iterationen mit sich, wodurch die Simulationszeit (Echtzeit) verlängert wird.

Eine Ausgabe der Anzahl der Iterationen zeigt $600$, wie erwartet ($\frac{600\,\si{\second}}{1\,\si{\second}/\text{step}}$). Im Gegensatz zum Ereignisbasierten Scheduler mit $5$ Iterationen liegt der Unterschied also bei einem Faktor von $120$.

Außerdem zeigt der zeitbasierte Ansatz (in diesem Fall), dass die meisten Iterationen keine Zustandsänderungen herbeiführen und damit nutzlos sind.

\paragraph{Simulationszeit von $1$ Woche:}
Eine Woche entspricht $604800$ Sekunden. Mittels \inlinecode{time()}-Funktion aus der \inlinecode{time} library in Python konnten die Ausführungszeiten beider Programme geprüft werden.

\[t_{\text{sched2p}} = 0.00652\,\si{\second}\]
\[t_{\text{timedriven}} = 0.182 \,\si{\second}\]
\[\frac{t_{\text{timedriven}}}{t_{\text{sched2p}}} \approx 30\]

\section{Teil B: 3-Phasen Scheduler}
\lstinputlisting[title={sched3p.py}]{code/3phase.py}

\begin{table}[H]
  \centering
\begin{tabular}{|c|c|c|l|}
\hline
\multicolumn{1}{|l|}{Simulationslauf} & \multicolumn{1}{l|}{Rot-Rot-Übergänge} & \multicolumn{1}{l|}{Rot-Grün-Übergänge} & Zustandsverlauf \\ \hline
1                                     & 1                                      & 1                                       & R-R-G-R         \\ \hline
2                                     & 0                                      & 2                                       & R-G-R-G-R       \\ \hline
3                                     & 1                                      & 1                                       & R-G-R-R         \\ \hline
4                                     & 1                                      & 1                                       & R-R-G-R         \\ \hline
5                                     & 2                                      & 0                                       & R-R-R-G         \\ \hline
6                                     & 1                                      & 1                                       & R-R-G-R         \\ \hline
7                                     & 1                                      & 1                                       & R-R-G-R         \\ \hline
8                                     & 3                                      & 0                                       & R-R-R-R         \\ \hline
9                                     & 0                                      & 2                                       & R-G-R-G-R       \\ \hline
10                                    & 1                                      & 1                                       & R-G-R-R         \\ \hline
\end{tabular}
\caption{Simulationsergebnis sched3p bei 10 Druchläufen mit Simulationszeit von $10\,\si{\minute}$}
\end{table}

\begin{table}[H]
\centering
\begin{tabular}{cc|c|c|}
\hline
\multicolumn{1}{|l|}{Simulationslauf} & \multicolumn{1}{l|}{Rot-Rot-Übergänge} & \multicolumn{1}{l|}{Rot-Grün-Übergänge} & \multicolumn{1}{l|}{Verhältnis (R-R/R-G)} \\ \hline
\multicolumn{1}{|c|}{1}               & 5851                                   & 5672                                    & 0.9694                                    \\ \hline
\multicolumn{1}{|c|}{2}               & 5732                                   & 5761                                    & 1.0051                                    \\ \hline
\multicolumn{1}{|c|}{3}               & 5695                                   & 5789                                    & 1.0165                                    \\ \hline
\multicolumn{1}{|c|}{4}               & 5681                                   & 5799                                    & 1.0208                                    \\ \hline
\multicolumn{1}{|c|}{5}               & 5749                                   & 5748                                    & 0.9998                                    \\ \hline
\multicolumn{1}{|c|}{6}               & 5729                                   & 5763                                    & 1.0059                                    \\ \hline
\multicolumn{1}{|c|}{7}               & 5743                                   & 5753                                    & 1.0017                                    \\ \hline
\multicolumn{1}{|c|}{8}               & 5745                                   & 5751                                    & 1.0010                                    \\ \hline
\multicolumn{1}{|c|}{9}               & 5733                                   & 5760                                    & 1.0047                                    \\ \hline
\multicolumn{1}{|c|}{10}              & 5732                                   & 5761                                    & 1.0051                                    \\ \hline
\multicolumn{1}{l}{}                  & \multicolumn{1}{l|}{}                  & Mittelwert                              & 1.003                                     \\ \cline{3-4}
\end{tabular}
\caption{Simulationsergebnis sched3p bei 10 Druchläufen mit Simulationszeit von $4$ Wochen}
\end{table}

\end{document}
