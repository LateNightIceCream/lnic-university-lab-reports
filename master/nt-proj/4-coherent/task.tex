\begin{taskspec}
HW4)\\

due May 11th 23:59 LT\\

1)\\
- create a long complex valued time series signal, perhaps a 2 or 4 state QAM\\
- the states should be persistent for 400 samples\\
- add complex noise to the time series\\

1a)\\
- derive a "signal to noise ratio"\\
- proposal: derive the deviation for each sample to the nominal QAM state position (e.g. 43+33j to 50+50j)\\
	-> sum of those distances in respect to the magnitute of the I/Q state\\

1b)\\
- apply coherent integrations: 1, 5, 10, 25, 50, 100\\
- plot the time series for 1, 25 and 100 CI (in one plot)\\

1c)\\
- investigate the effect of the diffent number of applied CI\\
-> plot "SNR" over CI, any mathematical approximation visible?\\

2) basically repeat the equivalent for incoherent integrations\\
- create a gaussian-like signal in the spectral domain and add noise\\

2a)\\
- apply 1, 5, 10, 25, 50, 100 incoherent integration\\
-> create different realisations of such spectra\\
-> do not split just one time series\\

2b)\\
- derive a "signal to noise ratio"\\
- proposal: sum the magnitudes of the spectra, first without noise, then for the 1..100 integrations\\
- plot the integrated spectrum for 1, 25 and 100 NCI\\

2c)\\
- investigate and plot the dependence of "SNR" to the number of integrated spectra (NCI)\\
-> any mathematical approximation visible?\\
\end{taskspec}
