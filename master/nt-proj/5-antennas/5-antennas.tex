\documentclass[10pt, a4paper]{article}

%
%   PACKAGES
%

\usepackage{graphicx}
\usepackage{amsmath}
\usepackage{amssymb}
\usepackage{mathtools}
\usepackage[binary-units=true]{siunitx}
\usepackage[table,xcdraw]{xcolor}
\usepackage{tikz}
\usepackage[most]{tcolorbox}
%\usepackage[english,german]{babel}
\usepackage[english]{babel}
\usepackage{blindtext}
\usepackage{import}
\usepackage{float}
\usepackage{varwidth}
\usepackage{esint}
\usepackage{subfig}
\usepackage[hyphens]{url}
\usepackage{hyperref}
\usepackage{pdfpages}
\usepackage{multicol}
\usepackage{listings}
\usepackage{geometry}
\usepackage{fontspec}
\usepackage[
    backend=biber,
]{biblatex}
\usepackage{todonotes}

%
%   GEOMETRY
%

\geometry{
  a4paper,
  %total={0.618\paperwidth,0.7639\paperheight},
  %left=0.190962\paperwidth,
  total={0.764\paperwidth,0.7639\paperheight},
  left=0.118\paperwidth,
}


%
%   FONT
%

%s\renewcommand{\familydefault}{\sfdefault}

%
%   COLORS
%

% \input{"/home/zamza/Documents/HS/Master/lnic-masters-protocols/preamble/colors.tex"}
\input{"colors.tex"}
\definecolor{yucky}{HTML}{dee2e6}
\definecolor{yuckytext}{HTML}{343a40}

%
%   HIGHLIGHTING
%
\newcommand{\minisec}[1]{\noindent\underline{\textbf{#1}}\\}

%
%   MATH
%

\newcommand\equalhat{\mathrel{\stackon[1.5pt]{=}{\stretchto{%
    \scalerel*[\widthof{=}]{\wedge}{\rule{1ex}{3ex}}}{0.5ex}}}}

\newcommand{\vecnabla}{\vec{\nabla}}
\newcommand{\rot}{\text{rot}\,}
\newcommand{\divv}{\text{div}\,}
\newcommand{\grad}{\text{grad}\,}
\newcommand{\divD}{\divv\vec{D}}
\newcommand{\divB}{\divv\vec{B}}
\newcommand{\divE}{\divv\vec{E}}
\newcommand{\rotD}{\rot\vec{D}}
\newcommand{\rotB}{\rot\vec{B}}
\newcommand{\rotE}{\rot\vec{E}}
\newcommand{\unitv}{\vec{e}}
\newcommand{\partialdev}[2]{\frac{\partial #1}{\partial #2}}

% ! equals
\newcommand{\hastobe}{\stackrel{!}{=}}

% dBm


%
%   HYPERLINKS
%
\hypersetup{
    colorlinks=true,
    linkcolor=blue-6,
    citecolor=blue-8,
    urlcolor=blue-6
}


%
%   FLOWCHART
%

\usetikzlibrary{shapes,arrows}

\tikzstyle{decision} = [diamond, draw, fill=blue!20,
    text width=4.5em, text badly centered, node distance=3cm, inner sep=0pt]
\tikzstyle{block} = [rectangle, draw=yucky, fill=yucky, text=yuckytext,
    text width=10em, text centered, rounded corners, minimum height=4em, minimum width=10em]
\tikzstyle{line} = [draw, -latex']
\tikzstyle{cloud} = [draw, ellipse,fill=red!20, node distance=3cm,
    minimum height=2em]

%
%   NOTE BOX
%

\colorlet{tcb_content_bg}{green-0}
\colorlet{tcb_title_bg}{green-0}
%\tcbset{colback=tcb_content_bg, colbacktitle=tcb_title_bg, colframe=tcb_content_bg, boxrule=0pt, bottomrule=0pt, frame hidden, sharp corners}

\tcbset{
  enhanced,
  sharp corners,
}

\newcommand{\notebox}[2]{
  \vspace{\baselineskip}
  \begin{tcolorbox}[title=#1]{#2}\end{tcolorbox}
  \vspace{\baselineskip}
}

\colorlet{zitat_bg}{gray-0}
\colorlet{zitat_strip}{gray-3}

\newtcolorbox{taskspec}[1][]{%
    colback=gray-1,
    fontupper=\selectfont\ttfamily,
    %grow to right by=-10mm,
    %grow to left by=-10mm,
    boxrule=0pt,
    boxsep=0pt,
    breakable,
    enhanced jigsaw,
    borderline west={1pt}{0pt}{gray-2},
    borderline east={1pt}{0pt}{gray-2},
    borderline north={1pt}{0pt}{gray-2},
    borderline south={1pt}{0pt}{gray-2},
    %colbacktitle={gray-2},
    %coltitle={gray-8},
    %fonttitle={\large\bfseries},
    attach title to upper={},
    #1,
}

\newtcolorbox{zitat}[2][]{%
    colback=zitat_bg,
    grow to right by=-10mm,
    grow to left by=-10mm,
    boxrule=0pt,
    boxsep=0pt,
    breakable,
    enhanced jigsaw,
    borderline west={4pt}{0pt}{zitat_strip},
    title={#2\par},
    colbacktitle={zitat_bg},
    coltitle={gray-8},
    fonttitle={\large\bfseries},
    attach title to upper={},
    #1,
}

\newcommand{\notebo}[2]{
  \vspace{\baselineskip}
  \begin{tcolorbox}[enhanced,
  sharp corners,
  boxrule=0pt,
  toptitle=0.1cm+1pt,%
  bottomtitle=-0.1cm+0.5em,%
  colframe=red-0,colback=red-0,coltitle=red-7,
  title style=red-0,
  fonttitle=\bfseries,fontupper=\normalsize,title=#1]{#2}\end{tcolorbox}
  \vspace{\baselineskip}
}

\newtcbtheorem[number within=chapter]{thm}{Theorem}{
  theorem style=change apart,
  enhanced,
  frame hidden,interior hidden,
  sharp corners,
  boxrule=0pt,
  left=0.2cm,right=0.2cm,top=0.2cm,
  toptitle=0.1cm+1pt,%        <-- I used your values here
  bottomtitle=-0.1cm+0.5em,%  <-- I used your values here
  colframe=white!25!black,colback=white,coltitle=white,
  title style=white!25!black,
  bottomrule=1pt,%  <-- reserve space
  borderline south={1pt}{0pt}{white!25!black},%---- draw line
  fonttitle=\bfseries,fontupper=\normalsize}{thm}


%
%   CODE
%

\definecolor{commentColor}{HTML}{adb5bd}
\definecolor{mygray}{rgb}{0.5,0.5,0.5}
\definecolor{stringColor}{HTML}{7048e8}
\definecolor{keywordColor}{HTML}{228be6}
\definecolor{backgroundColor}{HTML}{f1f3f5}
\definecolor{borderColor}{HTML}{f1f3f5}
\definecolor{inlineTextColor}{HTML}{495057}
\definecolor{leftRuleColor}{HTML}{868e96}
\definecolor{numbackgroundColor}{HTML}{f1f3f5}
\definecolor{numColor}{HTML}{adb5bd}

\newtcbox{\inlinebox}{enhanced,nobeforeafter,tcbox raise base,boxrule=0pt,top=0.062em,bottom=0.062em,
  right=0.382em,left=0.382em,arc=0.382em,boxsep=0.1em,before upper={\vphantom{dlg}},
  colframe=white,colback=backgroundColor}

\newcommand{\inlinecode}[1] {
  \inlinebox{\lstinline[language=Python, identifierstyle=\color{inlineTextColor}, basicstyle=\color{inlineTextColor}\ttfamily, keywordstyle=\color{inlineTextColor}]{#1}}
}

%\setmonofont{JetBrainsMono NF}[
%    Contextuals = Alternate,
%    Ligatures = TeX,
%]

\lstset{
  backgroundcolor=\color{backgroundColor},   % choose the background color; you must add \usepackage{color} or \usepackage{xcolor}; should come as last argument
  basicstyle=\small\ttfamily,        % the size of the fonts that are used for the code
  breaklines=true,                 % sets automatic line breaking
  commentstyle=\color{commentColor},    % comment style
  extendedchars=true,              % lets you use non-ASCII characters; for 8-bits encodings only, does not work with UTF-8
  firstnumber=1,                % start line enumeration with line 1000
  frame=single,	                   % adds a frame around the code
  frameshape={RYR}{Y}{Y}{RYR},
  keepspaces=true,                 % keeps spaces in text, useful for keeping indentation of code (possibly needs columns=flexible)
  keywordstyle=\color{keywordColor}\textbf,       % keyword style
  language=Python,                 % the language of the code
  morekeywords={*,...},            % if you want to add more keywords to the set
  numbers=left,                    % where to put the line-numbers; possible values are (none, left, right)
  numbersep=1.5em,                   % how far the line-numbers are from the code
  numberstyle=\tiny\color{commentColor}, % the style that is used for the line-numbers
  rulecolor=\color{borderColor},         % if not set, the frame-color may be changed on line-breaks within not-black text (e.g. comments (green here))
  showstringspaces=false,          % underline spaces within strings only
  showtabs=false,                  % show tabs within strings adding particular underscores
  stepnumber=1,                    % the step between two line-numbers. If it's 1, each line will be numbered
  stringstyle=\color{stringColor},     % string literal style
  tabsize=2,
  frame=l,
  framesep=2.12em,
  framexleftmargin=0em,
  fillcolor=\color{numbackgroundColor},
  rulecolor=\color{leftRuleColor},
  numberstyle=\ttfamily\tiny\color{numColor},
}

\newtcolorbox{codebg}[2][]{%
    colback=gray-1,
    %grow to right by=-10mm,
    %grow to left by=-10mm,
    boxrule=0pt,
    boxsep=0pt,
    breakable,
    enhanced jigsaw,
    rounded corners=east,
    arc=8pt,
    borderline west={1pt}{0pt}{gray-3},
    %title={#2\par},
    %colbacktitle={zitat_bg},
    bottomrule=0pt,
}


\newtcbox{\inlinecodee}{on line, boxrule=0pt, boxsep=0pt, top=2pt, left=2pt, bottom=2pt, right=2pt, colback=gray-2, colframe=white, fontupper={\ttfamily \footnotesize}}

\BeforeBeginEnvironment{minted}{\begin{codebg}}%
\AfterEndEnvironment{minted}{\end{codebg}}%

\BeforeBeginEnvironment{inputminted}{\begin{codebg}}%
\AfterEndEnvironment{inputminted}{\end{codebg}}%

\usepackage{minted}
\setminted{
  autogobble=true,
  breakautoindent=true,
  breaklines=true,
  escapeinside=§§,
  fontfamily=tt,
  fontsize=\footnotesize,
  frame=leftline,
  framerule=0pt,
  framesep=0.2em, % sufficient for up to 4 digits
  numbers=left,
  numbersep=0.2em,
  showspaces=false,
  showtabs=false,
  style=vs, % see: https://pygments.org/styles/
  tabsize=2,
  xleftmargin=1.5em,
  % colors
  bgcolor=gray-1,
}
\usemintedstyle{myown}

\tcbuselibrary{minted}

% minted line numbers
\renewcommand{\theFancyVerbLine}{\sffamily
\textcolor{gray-4}{\scriptsize
\oldstylenums{\arabic{FancyVerbLine}}}}

\renewcommand{\thempfootnote}{\arabic{mpfootnote}}

\begin{document}

\includepdf{./titlepage/titlepage.pdf}

\section*{Homework Tasks}
\begin{taskspec}
HW4)\\

due May 11th 23:59 LT\\

1)\\
- create a long complex valued time series signal, perhaps a 2 or 4 state QAM\\
- the states should be persistent for 400 samples\\
- add complex noise to the time series\\

1a)\\
- derive a "signal to noise ratio"\\
- proposal: derive the deviation for each sample to the nominal QAM state position (e.g. 43+33j to 50+50j)\\
	-> sum of those distances in respect to the magnitute of the I/Q state\\

1b)\\
- apply coherent integrations: 1, 5, 10, 25, 50, 100\\
- plot the time series for 1, 25 and 100 CI (in one plot)\\

1c)\\
- investigate the effect of the diffent number of applied CI\\
-> plot "SNR" over CI, any mathematical approximation visible?\\

2) basically repeat the equivalent for incoherent integrations\\
- create a gaussian-like signal in the spectral domain and add noise\\

2a)\\
- apply 1, 5, 10, 25, 50, 100 incoherent integration\\
-> create different realisations of such spectra\\
-> do not split just one time series\\

2b)\\
- derive a "signal to noise ratio"\\
- proposal: sum the magnitudes of the spectra, first without noise, then for the 1..100 integrations\\
- plot the integrated spectrum for 1, 25 and 100 NCI\\

2c)\\
- investigate and plot the dependence of "SNR" to the number of integrated spectra (NCI)\\
-> any mathematical approximation visible?\\
\end{taskspec}

\pagebreak

\section{Some Definitions}
%The following short definitions regarding antennas have been compiled using x and y \todo{cite a b}.

\begin{description}
\item[Antenna] A device which translates guided electromagnetic waves on a transmission line to waves in free space (and vice versa).
\item[Radiation Pattern $F$] The distribution of power (intensity) radiated from an antenna as a function of direction, in spherical coordinates (azimuth $\phi$, elevation $\theta$). Usually normalized to its maximum value.
\item[Directivity $D$] The power density as a function of direction normalized to the average power density of all directions. Antenna specifications typically give the directivity as peak directivity for all directions. It will determine how much more power the antenna radiates in its peak direction compared to an isotropic reference antenna.
\item[Gain $G$] Is similar to the directivity in that it describes the power density in the preak direction with reference to that of an isotropic antenna but it also takes into account losses. Especially relevant with electrically short antennas.
\item[Beamwidth] For directional antennas, the angle from the peak of the main lobe of the radiation pattern to the point at which the radiated power density decreases by a specified factor, typically $50\,\si{\percent}$ or $-3\,\si{\deci\bel}$ (Half Power Beam Width, HPBW).
\item[(Feedpoint-)Impedance] The relationship between the voltage and current at the input of the antenna (complex number). Typically for resonant antennas, the reactive part should be close to zero and the impedance should stay constant in the desired frequency band and matched to the feeding transmitter to allow for a low reflection coefficient.
\item[Polarization] Direction of the electric (and magnetic) field oscillation over time. If it stays constant, the field is said to be linearly polarized. The direction can also change if the electric field can be decomposed into components that are out of phase by $90\,\si{\degree}$. If the amplitude of these components is the same, the field is called cirularly polarized. If it is different, it is called eliptically polarized.
\item[Antenna Arrays] A collection of antennas that forms a combined radiation pattern. Often, the antennas can be driven individually to provide each with a different phase angle.
\end{description}



\pagebreak
\section{Antenna Types}

The following table \ref{tab:ant1} shows a few antenna examples with characteristic numbers.
Some radiation patterns for can also be seen in the figures below. Most of these characteristics are dependent upon the antenna geometry and are therefore only rough typical values.\\


\begin{table}[h]

\centering

\begin{minipage}{\textwidth}
\begin{tabular}{lcccccc} \toprule
    {Name} & {$D$ ($\si{\deci\bel}$)} & {$3\,\si{\deci\bel}$-Beamw.} & {Impedance} & {Bandwidth} & {Polarization} & {Typical Sizes} \\ \midrule\midrule
  {(nearly) omni-dir.} & & & & & & \\
  \midrule
    {$\lambda/2$ Dipole}  & 2.15 & $\approx 80\,\si{\degree}$ & $73\,\si{\ohm}$ & narrow & linear & $\lambda/2$  \\
  {$\lambda/4$ Monopole}  & 5.2  & $\approx 40\,\si{\degree}$ & $36.5\,\si{\ohm}$ & narrow & linear & $\lambda/4$ \\
  \midrule
  {directional} & & & & & & \\
  \midrule
    {Patch}  & 5{...}8  & typ. $65\,\si{\degree}$ & $50..300\,\si{\ohm}$ \footnote{depends on patch width}  & narrow & linear & some $10 \,\si{\milli\meter}$\\
    {Helix (Axial-Mode)}  & 8...14 & $\frac{5}{2}\frac{C}{\lambda}\sqrt{\frac{N\cdot S}{\lambda}}$  \footnote{$C$: circumference of one turn, $N$: number of turns, $S$: axial pitch between turns}  & $\approx 140\cdot C / \lambda$  & narrow & elliptical & $50\,\si{\milli\meter}$...few $\si{\meter}$\\
    {Horn}  & 10{...}20  & $11\,\si{\degree}...15\,\si{\degree}$  & $50\,\si{\ohm}$ \footnote{can be matched to waveguide geometry}  & wide & linear & $50...400\,\si{\mm}$ \\
    {Yagi}  & {...}20  & $30\,\si{\degree}...70\,\si{\degree}$ & $10...40\,\si{\ohm}$  & narrow & linear & $<10\,\si{\meter}$ \\
    {Parabolic}  & 10{...}40  & $\approx 70\lambda / D\,\si{\degree}$ \footnote{$D$ is the aperture diameter} & depends \footnote{depends on feed antenna}  & wide & depends \footnote{depends on feed antenna}  & $0.5...500\,\si{\meter}$ \footnote{see FAST radio telescope} \\ \midrule
  {antenna groups} & & & & & & \\
  \midrule
    {2x2 Patch Array}  & > patch  & $\approx 15\,\si{\degree}$ & $50\,\si{\ohm}$ & narrow  & linear &  some $10\,\si{\milli\meter}$\\
    {Collinear Dipoles}  & $10\cdot\log{n}$ \footnote{$n:$ number of elements, gain relative to single element}  & $\propto 1/n$ & $73\,\si{\ohm} / n$ & narrow & linear & $n \cdot (\lambda / 2 + s)$ \\
    \bottomrule
\end{tabular}
\end{minipage}

\caption{Compilation of different antenna types.}\label{tab:ant1}

\end{table}

% https://en.wikipedia.org/wiki/Parabolic_antenna
% https://en.wikipedia.org/wiki/Collinear_antenna_array
% https://en.wikipedia.org/wiki/Monopole_antenna


\begin{table}[h]

\centering

\begin{minipage}{\textwidth}
  \centering
\begin{tabular}{p{0.15\linewidth}p{0.6\linewidth}} \toprule
    {Name} & Typical Uses\\ \midrule\midrule
    Dipole & fixed-frequency, omnidirectional applicatins, e.g. WiFi routers \\ \midrule
    Monopole & mobile devices, vehicles\\ \midrule
    Patch & PCBs and ICs\\ \midrule
    Helix & where circular polarization is needed, e.g. certain space communication\\ \midrule
    Horn & waveguides, feed for parabolic antennas, low-loss applications\\\midrule
    Yagi & terrestrial communication, fixed-frequency applications \\\midrule
    Parabolic & satellite communication links, radio telescopes\\\midrule
    Arrays & interferometry, beamsteering applications\\
    \bottomrule
\end{tabular}
\end{minipage}

\caption{Typical antenna application areas.}\label{tab:ant2}

\end{table}

% https://en.wikipedia.org/wiki/Parabolic_antenna
% https://en.wikipedia.org/wiki/Collinear_antenna_array
% https://en.wikipedia.org/wiki/Monopole_antenna


\begin{figure}[H]
  \begin{minipage}[t]{0.45\textwidth}
    \centering
    \includegraphics[width=\textwidth]{graphics/dipole_theta.pdf}
    \caption{$\lambda/4$ Dipole characteristic in the elevation plane. Antenna axis lies vertically.}
  \end{minipage}\hfill
  \begin{minipage}[t]{0.45\textwidth}
    \centering
    \includegraphics[width=\textwidth]{graphics/helical_theta.pdf}
    \caption{Helical characteristic in the elevation plane. Separation: $0.22\,\si{\meter}$, Number of turns: $10$, Frequency: $500\,\si{\mega\hertz}$ ($\lambda=0.6\,\si{\meter}$)}
  \end{minipage}
\end{figure}



\begin{figure}[H]
  \begin{minipage}[t]{0.45\textwidth}
    \centering
    \includegraphics[width=\textwidth]{graphics/patch_theta.pdf}
    \caption{Single Patch characteristic in the elevation plane for two different $\phi$ angles. $W=L=0.5\,\si{\lambda}$.}
  \end{minipage}\hfill
  \begin{minipage}[t]{0.45\textwidth}
    \centering
    \includegraphics[width=\textwidth]{graphics/dipole_array_theta.pdf}
    \caption{Characteristic of equally spaced linear dipole array in the elevation plane. $N=8$, Separation: $\lambda/2$}
  \end{minipage}
\end{figure}


\begin{figure}[H]
  \begin{minipage}[t]{0.45\textwidth}
 \centering
  \includegraphics[width=\textwidth]{graphics/yagi.png}
  \caption{Example of a Yagi-Uda characterisitic. Taken from \url{https://www.researchgate.net/figure/Radiation-patterns-of-two-Yagi-Uda-antennas-a-with-high-gain-and-b-with-low-gain_fig2_363745716} under CC BY-NC-ND 4.0}

  \end{minipage}\hfill
  \begin{minipage}[t]{0.45\textwidth}
  \centering
  \includegraphics[width=\textwidth]{graphics/yagi2.png}
  \caption{Drawing of a typical Yagi-Uda antenna.}
  \end{minipage}
\end{figure}


\begin{figure}[H]
  \begin{minipage}[t]{0.45\textwidth}
    \centering
    \includegraphics[width=\textwidth]{graphics/parabolic_theta.pdf}
    \caption{Example of Parabolic characteristic for $D=10\lambda$.}
  \end{minipage}\hfill
  \begin{minipage}[t]{0.45\textwidth}
    \centering
    \includegraphics[width=\textwidth]{graphics/parab.jpg}
    \caption{Satellite communications parabolic antenna. Taken from \url{https://en.wikipedia.org/wiki/Parabolic_antenna\#/media/File:Erdfunkstelle_Raisting_2.jpg} under CC BY-SA 2.5}
   \end{minipage}
\end{figure}



% ------------------------------------------------------------------------------

\printbibheading
\begin{refsection}[sources.bib]
\nocite{*}
\printbibliography[heading=subbibliography,title={Literature}]
\end{refsection}

\begin{refsection}[web.bib]
\nocite{*}
\printbibliography[heading=subbibliography,title={Web}]
\end{refsection}

% \begin{refsection}[software.bib]
% \nocite{*}
% \printbibliography[heading=subbibliography,title={Software Used}]
% \end{refsection}

\end{document}
