%The following short definitions regarding antennas have been compiled using x and y \todo{cite a b}.

\begin{description}
\item[Antenna] A device which translates guided electromagnetic waves on a transmission line to waves in free space (and vice versa).
\item[Radiation Pattern $F$] The distribution of power (intensity) radiated from an antenna as a function of direction, in spherical coordinates (azimuth $\phi$, elevation $\theta$). Usually normalized to its maximum value.
\item[Directivity $D$] The power density as a function of direction normalized to the average power density of all directions. Antenna specifications typically give the directivity as peak directivity for all directions. It will determine how much more power the antenna radiates in its peak direction compared to an isotropic reference antenna.
\item[Gain $G$] Is similar to the directivity in that it describes the power density in the preak direction with reference to that of an isotropic antenna but it also takes into account losses. Especially relevant with electrically short antennas.
\item[Beamwidth] For directional antennas, the angle from the peak of the main lobe of the radiation pattern to the point at which the radiated power density decreases by a specified factor, typically $50\,\si{\percent}$ or $-3\,\si{\deci\bel}$ (Half Power Beam Width, HPBW).
\item[(Feedpoint-)Impedance] The relationship between the voltage and current at the input of the antenna (complex number). Typically for resonant antennas, the reactive part should be close to zero and the impedance should stay constant in the desired frequency band and matched to the feeding transmitter to allow for a low reflection coefficient.
\item[Polarization] Direction of the electric (and magnetic) field oscillation over time. If it stays constant, the field is said to be linearly polarized. The direction can also change if the electric field can be decomposed into components that are out of phase by $90\,\si{\degree}$. If the amplitude of these components is the same, the field is called cirularly polarized. If it is different, it is called eliptically polarized.
\item[Antenna Arrays] A collection of antennas that forms a combined radiation pattern. Often, the antennas can be driven individually to provide each with a different phase angle.
\end{description}

