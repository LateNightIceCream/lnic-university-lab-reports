\subsection{Halb- und Volladdierer}

Addierschaltungen werden genutzt, um binäre Arithmetik mittels kombinatorischer Logik umzusetzen. Die beiden grundlegenden Bausteine sind Halb- und Volladdierer.

\subsubsection{Halbaddierer}
Ein Halbaddierer erzeugt das Summen- sowie das Übertragsbit (Carry) der Addition seiner zwei Eingänge $A$ und $B$.
Die Summe kann über ein XOR-Gate und das Carry-Bit über ein einfaches AND-Gate realisiert werden.


\begin{table}[H]
\centering
 \begin{tabular}{|c|c|c|c|}
  \hline
    $A$ & $B$ & $OUT$ & $C_{OUT}$ \\\hline
    0 & 0 & 0 & 0\\\hline
    0 & 1 & 1 & 0\\\hline
    1 & 0 & 1 & 0\\\hline
    1 & 1 & 0 & 1\\\hline
  \end{tabular}
  \caption{Wahrheitstabelle des Halbaddierers.}
\end{table}

\begin{figure}[H]
  \def\svgwidth{0.382\textwidth}\normalsize
  \begin{center}
     \import{graphics/vorbereitung/}{halfadder.pdf_tex}
  \end{center}
  \caption{Schaltbild des Halbaddierers.}
  \label{fig:halfadder}
\end{figure}


\subsubsection{Volladdierer}
Zusätzlich zu den Ein- und Ausgängen des Halbaddierers besitzt ein Volladdierer ein Übertragsbit als Eingang. Die Schaltfunktion für das Summenbit des Volladdierers ist
\[
  OUT = A \oplus B \oplus C_{in}
\]
\[
  C_{out} = A \land B \lor (A \oplus B) \land C_{in}
\]

Dies ermöglicht es, einen Volladdierer aus zwei Halbaddierern aufzubauen.

\begin{figure}[H]
  \def\svgwidth{\textwidth}\normalsize
  \begin{center}
     \import{graphics/vorbereitung/}{halfadder-fulladder.pdf_tex}
  \end{center}
  \caption{Schaltbild des Volladdierers aus zwei Halbaddierern.}
  \label{fig:fulladder-1}
\end{figure}
