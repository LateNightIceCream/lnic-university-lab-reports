Temperaturänderungen stellen für den Transistor als Halbleiterbauelement eine
externe Energiezufuhr und damit eine Störung des thermodynamischen
Gleichgewichts dar. Die Ladungsträgerdichten der einzelnen Bereiche
erhöhen sich, die Weiten der Raumladungszonen verringern sich und der Transistor
wird insgesamt leitfähiger. Dadurch erhöht sich auch der
Stromverstärkungsfaktor $\beta$, was z.B. den Arbeitspunkt, der bei der
Schaltungsdimensionierung angenommen wurde, verschieben kann. Da dieser zusätzlich
fertigungsbedingt abweichen kann, strebt man einen
Arbeitspunkt an, der möglichst unabhängig von der Stromverstärkung ist. Dies
wird z.B. durch die Arbeitspunkteinstellung mit einem 4-Widerstandsnetzwerk mit Stromgegenkopplung oder die Einstellung des Emitterstroms durch eine Stromquelle erreicht.

