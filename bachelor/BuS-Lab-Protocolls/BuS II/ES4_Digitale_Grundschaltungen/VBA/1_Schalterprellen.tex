Schalterprellen oder auch Kontaktprellen bezeichnet den Effekt des mehrfachen
Öffnens bzw. Schließens bei einmaligem Auslösen eines mechanischen
Schalters/Tasters, ausgelöst durch die Federeigenschaften der Kontaktmetalle.

In Logikschaltkreisen stellt es ein Problem dar, da es den im Tasterdruck codierten
Logikzustand (HIGH oder LOW) durch eine schnelle Folge von wechselnden Zuständen
vor dem eigentlich gewollten Zustand behindert.

Um die Auswirkungen zu verhindern und möglichst eindeutige Zustände zu erhalten,
kann man Entprellmaßnahmen einführen. Eine Möglichkeit ist eine
Monoflopschaltung, die durch den Taster ausgelöst wird. Dadurch wird beim ersten
erkannten Zustandswechsel des Tasters für eine durch die Schaltung definierte
Zeit der entsprechende Zustand am Ausgang des Monoflops durchgeschalten wodurch
weitere, folgende Prellimpulse keine Auswirkungen auf den Ausgangszustand haben.

Weitere Möglichkeiten sind:
\begin{itemize}
\item SR-Flip-Flop (SPDT-Schalter)
\item RC-Tiefpass (+Schmitt-Trigger)
\item Softwarelösungen, z.B. bei Microcontrollern
\end{itemize}

