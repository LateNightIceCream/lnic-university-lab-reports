%-------------------------------------------------------------------------------
\subsection{RFC 1918}

Der RFC (Request for Comments) 1918: \emph{Address Allocation for private
 Internets} definiert die IP-Adresszuweisung für den privaten Adressbereich.
Dies erlaubt zwei Hosts gleiche IP-Adressen zu besitzen, sofern diese voneinander
isoliert sind (private Netzwerke). Zwischen beiden privaten Netzwerken stehen
dann die NAT/PAT-Devices, welche für die Übersetzung in den öffentlichen
Adressbereich und zurück sorgen.\\

Entscheidendes Kriterium ist, dass IP-Adressen, die aus dem, nach RFC 1918
definierten, privaten Adressbereich stammen auch nur in internen, privaten
Netzwerken laufen können. Das bedeutet, dass Hosts mit privaten IP-Adressen
nicht aus dem Internet erreicht werden können.\\

Es gibt 3 verschiedene Adressbereiche, welche unterschiedliche Anzahlen an Bits,
ausgehend von einer Grundadresse, für den privaten Bereich reservieren (vgl.
Netzklassen) (nicht zu verwechseln mit Subnetting).

\begin{table}[H]
 \begin{center}
\begin{tabular}{llll}
\rowcolor{teal-0}
Adressbereich                 & Blockgröße & CIDR Kennz.    & Anzahl\\
10.0.0.0 - 10.255.255.255     & 24 bit     & 10.0.0.0/8     & 16777216\\
172.16.0.0 - 172.31.255.255   & 20 bit     & 176.16.0.0/12  & 1058576\\
192.168.0.0 - 192.168.255.255 & 16 bit     & 192.168.0.0/16 & 65536
\end{tabular}
 \end{center}
\caption{RFC 1918 private Adresszuweisungen}
\end{table}

Die restlichen $2^{32}-2^{24}-2^{20}-2^{16} = 4.277.075.968$ Adressen für den
öffentlichen Bereich werden von der \textsc{IANA} (Internet Assigned Numbers
Authority) verwaltet.

%-------------------------------------------------------------------------------
\subsection{Anzahl intern local auf intern global}
PAT weist jeder TCP oder UDP Kommunikation eine einzigartige Quellportnummer zu,
was die Abbildung der Adressen mehrerer lokaler (insidel local) Geräte auf
eine einzige öffentliche (inside global) Adresse ermöglicht.\\

Da der Port eine 16-bit Zahl ist und somit maximal den Wert 65535 haben kann,
ist die maximale Anzahl dieser Zuweisungen auf eine einzige inside global IP
durch die Portnummer limitiert.\\

Hinzu kommt, dass ein Host-Gerät (einzelne inside local IP) durchaus
gleichzeitig mehrere Verbindungen mit unterschiedlichen Quellports herstellen
kann (z.B. Browser bei Öffnen einer Website). Dadurch wird die Anzahl der
mit NAT verwalteten Geräte limitiert.

%-------------------------------------------------------------------------------
\subsection{Vor- und Nachteile von NAT / PAT}

\begin{table}[H]
\begin{tabularx}{\textwidth}{|X|X|}
  \textbf{Vorteile} & \textbf{Nachteile}\\
  \hline
  $\rightarrow$ Mehrere Hosts auf eine IP-Adresse zuordenbar, dadurch werden IPv4 Adressen gespart & $\rightarrow$ Erhöhter technischer Aufwand / zusätzlicher Schritt der Übersetzung notwendig\\
  $\rightarrow$ Anonymisierung der Hosts gegenüber der outside global/local Seite &$\rightarrow$ Begrenzter Speicher des NAT-Gerätes bei hoher Anzahl an Übersetzungseinträgen\\
  $\rightarrow$ Höhere Flexibilität und Übersichtlichkeit im Network-Design & $\rightarrow$ Komplikationen bei VPN\\
  & $\rightarrow$ Router sollte eigentlich nicht auf Ebene 4 (Portnummern) agieren
\end{tabularx}
\end{table}

%-------------------------------------------------------------------------------
\subsection{Unterscheidungsmerkmale am NAT-Device}
Die Unterscheidungsmerkmale sind:

\[\{\text{Quell-IP, Quell-Port, Ziel-IP, Ziel-Port, Protokoll}\}\]\\

Das NAT-Device (i.d.R. Router) führt eine Tabelle über bestehende Verbindungen
und den zugehörigen Adress- und Portübersetzungen.\\

Ein Sonderfall bildet das ICMP-Protokoll (z.B. bei ping-Befehl), da es auf
Ebene 3 arbeitet und daher keine Portnummern verwendet. Dieser Fall wird im RFC
5508 definiert. Für ICMP-Anfrage und -Antwort Nachrichten (query/reply) legt
das NAT-Gerät, ähnlich wie bei PAT, eine Query-ID an, die den jeweiligen Host
identifiziert.\\

