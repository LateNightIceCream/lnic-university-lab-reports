\subsubsection{Einfügedämpfung}
Die Einfügedämpfung ist das Verhältnis der Leistung, die an einer Last über z.B.
eine Leitung umgesetzt wird, zu der Leistung, die
\emph{direkt} an der Last umgesetzt werden würde (ohne die Leitung). Sie gibt
also den Leistungsverlust bzw. die Dämpfung an, die das Signal durch das \textit{Einfügen} der Leitung in den
Signalweg erfährt.

\subsubsection{Verstärkung}
Die Verstärkung ist allgemein das Verhältnis von Ausgangs- zu Eingangsgröße,
also z.B. Aus- zu Eingangsspannung. Ist die Verstärkung kleiner eins, handelt es
sich um eine Dämpfung. Logarithmisch/im Dezibel-Maß kann sie abhängig von der
betrachtetn Größe dargestellt werden durch
\[V = 10 \cdot \log_{10}{\frac{P_2}{P_1}}\]
wenn es sich um eine Leistung handelt oder
\[V = 20 \cdot \log_{10}{\frac{U_2}{U_1}}\]
wenn es sich um eine Spannung handelt.

\subsubsection{Reflexionsdämpfung}
Die Reflexionsdämpfung $R$ ist die Dämpfung der reflektierten Welle.
\[R = \frac{P_\mathrm{reflektiert}}{P_\mathrm{ein}}\]
Sie entspricht dem inversen Reflexionsfaktor.


\subsubsection{S-Parameter}
Die S(treu)-Parameter (vom engl. \emph{Scattering-Parameter}) sind eine Reihe an
komplexwertigen Parametern (Betrag + Phase) zur Charakterisierung von n-Toren,
in der Regel Vierpole.

Bei der Berechnung der S-Parameter bezieht man sich auf ein Vierpolmodell, bei
welchem $a_1$ und $a_2$ die auf das jeweilige Tor zulaufenden und $b_1$ und
$b_2$ die vom jeweiligen Tor ablaufenden Wellengrößen sind (Tor 1/Tor 2).

Bei einem Vierpolnetzwerk existieren die folgenden S-Parameter:
\begin{description}
  \item[$S_{11}$, der Reflexionskoeffizient, primärseitig,] beschreibt das
    Reflexions-zu-Transmissions-Verhältnis von Spannungs- bzw. Stromwellen.
    \[S_{11} = \frac{b_1}{a_1}\]
    \item[$S_{21}$, der Transmissionskoeffizient, primärseitig,] beschreibt das
      Verhältnis der am Tor 2 auslaufenden Welle zur am Tor 1 einlaufenden
      Welle. Über der Frequenz also die Übertragungsfunktion. (\glqq Ausgang zu
      Eingang\grqq)
      \[S_{21} = \frac{b_2}{a_1}\]
    \item[$S_{22}$, der Reflexionskoeffizient, sekundärseitig]
      Wie $S_{11}$ nur vom anderen Tor blickend.
      \[S_{22} = \frac{b_2}{a_2}\]
    \item[$S_{12}$, der Transmissionskoeffizient, sekundärseitig]
      Wie $S_{21}$ nur vom anderen Tor blickend.
      \[S_{12} = \frac{b_1}{a_2}\]
\end{description}

Der erste Index steht jeweils für den Wirkungsort, der zweite für die Ursache.
