Bei einer Messung treten gewöhnlicherweise systematische Fehler auf. Diese sind
durch die Messeinrichtung bedingt und müssen vor der Messung erfasst und
korrigiert werden, um möglichst genaue Messergebnisse erzielen zu können. Diese
Störeinflüsse treten z.B. durch die ungewollten Eigenschaften von Messleitungen
auf, da sie eigene Übertragungsfunktionen und Reflexionsfaktoren besitzen,
welche das Messergebnis verfälschen.

Bei der \emph{Kalibrierung} wird die Abweichung der Messeinrichtung mit einem
Referenzobjekt (Standard, Normal) verglichen. Beim Netzwerkanalysator ist dies
typischerweise ein \glqq Short, Open, Load\grqq (SOL) bzw. \glqq Short, Open,
Load, Through\grqq (SOLT), welches die Messleitung (Kabel, Verbindungen) mit den jeweiligen Impedanzen
abschließt (Mech-Cal). Diese werden nacheinander mit den verwendeten Ports des NWAs
verbunden. Die Kalibrierung kann ebenfalls mit einem automatischen
Kalibrierungsmodul (eCal) durchgeführt werden, welches für jeden Port die
entsprechenden Kalibrierungsbedingungen automatisch bereitstellt. Diese Methode
verringert die mechanische Belastung der Verbindungen und ist, gerade bei 4-Port-Kalibrierung, schneller.

Vor der Kalibrierung des NWAs muss der zu messende Frequenzbereich eingestellt
werden (start \& stop frequency). Bei einer Umstellung des Frequenzbereiches muss
erneut kalibriert werden.