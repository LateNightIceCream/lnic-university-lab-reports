Ein Netzwerkanalysator misst die Netzwerkparameter elektrischer Netzwerke,
insbesondere Vierpoleigenschaften, wie
\begin{itemize}
  \item Impedanz,
  \item Reflexionsfaktor,
  \item Transmission (Übertragungsfunktion),
\end{itemize}
deren Werte als Frequenzgänge auf einem Bildschirm dargestellt werden.

Ein skalarer Netzwerkanalysator (SNA) misst nur die Beträge der entsprechenden
Größen, ein vektorieller Netzwerkanalysator (VNA)
hingegen misst Beträge \textbf{und} Phasen, also \emph{komplexe} Werte (der Begriff
\emph{vektoriell} ist daher etwas irreführend).
