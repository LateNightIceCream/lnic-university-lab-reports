\documentclass[11pt, a4paper]{article}

%%% SST LAB PROTOCOLL PREAMBLE
%%% 2019
%%%%%%%%%%%%%%%%%%%%%%%%%%%%%%%


%%% PACKAGES
%%%%%%%%%%%%%%%%%%%%%%%%%%%

\usepackage[ngerman]{babel}

\usepackage[utf8]{inputenc}
\usepackage{amsmath}
\usepackage{pgfplots}
\usepackage{tikz}
\usepackage[many]{tcolorbox}
\usepackage{graphicx}
\graphicspath{ {./graphics/} }
\usepackage{pdfpages}
\usepackage{dashrule}
\usepackage{float}
\usepackage{siunitx}
\usepackage{trfsigns}
\usepackage{booktabs}
\usepackage[european]{circuitikz}
\usepackage{tcolorbox}

%%% DOCUMENT GEOMETRY
%%%%%%%%%%%%%%%%%%%%%%%%%%%

\usepackage{geometry}
\geometry{
 a4paper,
 total={0.6180339887498948\paperwidth,0.6180339887498948\paperheight},
 top = 0.1458980337503154\paperheight,
 bottom = 0.1458980337503154\paperheight
 }
\setlength{\jot}{0.013155617496424828\paperheight}
\linespread{1.1458980337503154}

\setlength{\parskip}{0.013155617496424828\paperheight} % paragraph spacing


%%% COLORS
%%%%%%%%%%%%%%%%%%%%%%%%%%%

\definecolor{red1}{HTML}{f38181}
\definecolor{yellow1}{HTML}{fce38a}
\definecolor{green1}{HTML}{95e1d3}
\definecolor{blue1}{HTML}{66bfbf}
\definecolor{hsblue}{HTML}{00b1db}
\definecolor{hsgrey}{HTML}{afafaf}

%%% CONSTANTS
%%%%%%%%%%%%%%%%%%%%%%%%%%%
\newlength{\smallvert}
\setlength{\smallvert}{0.0131556\paperheight}


%%% COMMANDS
%%%%%%%%%%%%%%%%%%%%%%%%%%%

% differential d
\newcommand*\dif{\mathop{}\!\mathrm{d}}

% horizontal line
\newcommand{\holine}[1]{
  	\begin{center}
	  	\noindent{\color{hsgrey}\hdashrule[0ex]{#1}{1pt}{3mm}}\\%[0.0131556\paperheight]
  	\end{center}
}

% mini section
\newcommand{\minisec}[1]{ \noindent\underline{\textit {#1} } \\}

% quick function plot
\newcommand{\plotfun}[3]{
  \vspace{0.021286\paperheight}
  \begin{center}
    \begin{tikzpicture}
      \begin{axis}[
        axis x line=center,
        axis y line=center,
        ]
        \addplot[draw=red1][domain=#2:#3]{#1};
      \end{axis}
    \end{tikzpicture}
  \end{center}
}

% box for notes
\newcommand{\notebox}[1]{

\tcbset{colback=white,colframe=green1!100!black,title=Note!,width=0.618\paperwidth,arc=0pt}

 \begin{center}
  \begin{tcolorbox}[]
   #1 
  \end{tcolorbox}
 
 \end{center} 
 
}

% box for equation
\newcommand{\eqbox}[2]{
	
	\tcbset{colback=white,colframe=green1!100!black,title=,width=#2,arc=0pt}
	
	\begin{center}
		\begin{tcolorbox}[ams align*]
				#1
		\end{tcolorbox}
		
	\end{center} 
	
}
% END OF PREAMBLE

\begin{document}

\begin{center}
  \Large{Echtzeit- und Netzwerkprogrammierung: Einführung}
\end{center}

\begin{flushright}
  R. Grünert\\
  \today
\end{flushright}

\section{Betriebssysteme}
Das Betriebssystem ist eine Schicht unterhalb der Anwendungssoftware (Browser, Editor, Explorer, Terminal, etc.), die dessen Ablauf steuert (CPU Scheduling, Speicherverwaltung) und die Kommunikation mit der Hardware verwaltet (Treiber, etc.).\\

Das Betriebssystem muss den Anwendungen \glqq Abstraktionen\grqq{} zur Verfügung stellen, die den Zugriff auf die Hardware für die Anwendungssoftware \emph{abstrahieren}.

\subsection{Prozessverwaltung}
\begin{figure}[H]
\centering
\resizebox{0.618\textwidth}{!}{\import{graphics/}{prozessverwaltung.pdf_tex}}
\end{figure}

Für jedes Programm (besser: jeder Prozess: Prozess ist ein Programm in Ausführung), das vom Betriebssystem verwaltet wird gibt es einen \emph{Kontext}, welcher die für das Programm zugewiesenen Speicherbereiche (boundaries Lo, Hi), Instructionpointer (IP) und Stackpointer (SP) speichert, um den Mehrprogrammbetrieb zu ermöglichen.

Für die Prozessorzuweisung können Prozesse die Zustände Running, Ready und Blocked (z.B. wenn gerade auf I/O Interrupt gewartet wird) vom OS zugewiesen werden.

\section{Speicherbelegung eines Prozesses}
Jedem Prozess wird Speicher idR. wie folgt zugewiesen:

\begin{figure}[H]
\centering
\resizebox{\textwidth}{!}{\import{graphics/}{prozessspeicher.pdf_tex}}
\end{figure}

Indem man die \emph{Program Break} hoch- oder herunterbewegt, ändert man den Speicher der dem Prozess zugewiesen ist. Der syscall unter Linux dafür ist \emph{brk()} (wird auch von \emph{malloc()} verwendet!).


%\notebox{hey}{\blindtext}
\end{document}
