\documentclass[11pt, a4paper]{article}

%
%   PACKAGES
%

\usepackage{graphicx}
\usepackage{amsmath}
\usepackage{amssymb}
\usepackage{mathtools}
\usepackage[binary-units=true]{siunitx}
\usepackage[table,xcdraw]{xcolor}
\usepackage{tikz}
\usepackage[most]{tcolorbox}
%\usepackage[english,german]{babel}
\usepackage[english]{babel}
\usepackage{blindtext}
\usepackage{import}
\usepackage{float}
\usepackage{varwidth}
\usepackage{esint}
\usepackage{subfig}
\usepackage[hyphens]{url}
\usepackage{hyperref}
\usepackage{pdfpages}
\usepackage{multicol}
\usepackage{listings}
\usepackage{geometry}
\usepackage{fontspec}
\usepackage[
    backend=biber,
]{biblatex}
\usepackage{todonotes}

%
%   GEOMETRY
%

\geometry{
  a4paper,
  %total={0.618\paperwidth,0.7639\paperheight},
  %left=0.190962\paperwidth,
  total={0.764\paperwidth,0.7639\paperheight},
  left=0.118\paperwidth,
}


%
%   FONT
%

%s\renewcommand{\familydefault}{\sfdefault}

%
%   COLORS
%

% \input{"/home/zamza/Documents/HS/Master/lnic-masters-protocols/preamble/colors.tex"}
\input{"colors.tex"}
\definecolor{yucky}{HTML}{dee2e6}
\definecolor{yuckytext}{HTML}{343a40}

%
%   HIGHLIGHTING
%
\newcommand{\minisec}[1]{\noindent\underline{\textbf{#1}}\\}

%
%   MATH
%

\newcommand\equalhat{\mathrel{\stackon[1.5pt]{=}{\stretchto{%
    \scalerel*[\widthof{=}]{\wedge}{\rule{1ex}{3ex}}}{0.5ex}}}}

\newcommand{\vecnabla}{\vec{\nabla}}
\newcommand{\rot}{\text{rot}\,}
\newcommand{\divv}{\text{div}\,}
\newcommand{\grad}{\text{grad}\,}
\newcommand{\divD}{\divv\vec{D}}
\newcommand{\divB}{\divv\vec{B}}
\newcommand{\divE}{\divv\vec{E}}
\newcommand{\rotD}{\rot\vec{D}}
\newcommand{\rotB}{\rot\vec{B}}
\newcommand{\rotE}{\rot\vec{E}}
\newcommand{\unitv}{\vec{e}}
\newcommand{\partialdev}[2]{\frac{\partial #1}{\partial #2}}

% ! equals
\newcommand{\hastobe}{\stackrel{!}{=}}

% dBm


%
%   HYPERLINKS
%
\hypersetup{
    colorlinks=true,
    linkcolor=blue-6,
    citecolor=blue-8,
    urlcolor=blue-6
}


%
%   FLOWCHART
%

\usetikzlibrary{shapes,arrows}

\tikzstyle{decision} = [diamond, draw, fill=blue!20,
    text width=4.5em, text badly centered, node distance=3cm, inner sep=0pt]
\tikzstyle{block} = [rectangle, draw=yucky, fill=yucky, text=yuckytext,
    text width=10em, text centered, rounded corners, minimum height=4em, minimum width=10em]
\tikzstyle{line} = [draw, -latex']
\tikzstyle{cloud} = [draw, ellipse,fill=red!20, node distance=3cm,
    minimum height=2em]

%
%   NOTE BOX
%

\colorlet{tcb_content_bg}{green-0}
\colorlet{tcb_title_bg}{green-0}
%\tcbset{colback=tcb_content_bg, colbacktitle=tcb_title_bg, colframe=tcb_content_bg, boxrule=0pt, bottomrule=0pt, frame hidden, sharp corners}

\tcbset{
  enhanced,
  sharp corners,
}

\newcommand{\notebox}[2]{
  \vspace{\baselineskip}
  \begin{tcolorbox}[title=#1]{#2}\end{tcolorbox}
  \vspace{\baselineskip}
}

\colorlet{zitat_bg}{gray-0}
\colorlet{zitat_strip}{gray-3}

\newtcolorbox{taskspec}[1][]{%
    colback=gray-1,
    fontupper=\selectfont\ttfamily,
    %grow to right by=-10mm,
    %grow to left by=-10mm,
    boxrule=0pt,
    boxsep=0pt,
    breakable,
    enhanced jigsaw,
    borderline west={1pt}{0pt}{gray-2},
    borderline east={1pt}{0pt}{gray-2},
    borderline north={1pt}{0pt}{gray-2},
    borderline south={1pt}{0pt}{gray-2},
    %colbacktitle={gray-2},
    %coltitle={gray-8},
    %fonttitle={\large\bfseries},
    attach title to upper={},
    #1,
}

\newtcolorbox{zitat}[2][]{%
    colback=zitat_bg,
    grow to right by=-10mm,
    grow to left by=-10mm,
    boxrule=0pt,
    boxsep=0pt,
    breakable,
    enhanced jigsaw,
    borderline west={4pt}{0pt}{zitat_strip},
    title={#2\par},
    colbacktitle={zitat_bg},
    coltitle={gray-8},
    fonttitle={\large\bfseries},
    attach title to upper={},
    #1,
}

\newcommand{\notebo}[2]{
  \vspace{\baselineskip}
  \begin{tcolorbox}[enhanced,
  sharp corners,
  boxrule=0pt,
  toptitle=0.1cm+1pt,%
  bottomtitle=-0.1cm+0.5em,%
  colframe=red-0,colback=red-0,coltitle=red-7,
  title style=red-0,
  fonttitle=\bfseries,fontupper=\normalsize,title=#1]{#2}\end{tcolorbox}
  \vspace{\baselineskip}
}

\newtcbtheorem[number within=chapter]{thm}{Theorem}{
  theorem style=change apart,
  enhanced,
  frame hidden,interior hidden,
  sharp corners,
  boxrule=0pt,
  left=0.2cm,right=0.2cm,top=0.2cm,
  toptitle=0.1cm+1pt,%        <-- I used your values here
  bottomtitle=-0.1cm+0.5em,%  <-- I used your values here
  colframe=white!25!black,colback=white,coltitle=white,
  title style=white!25!black,
  bottomrule=1pt,%  <-- reserve space
  borderline south={1pt}{0pt}{white!25!black},%---- draw line
  fonttitle=\bfseries,fontupper=\normalsize}{thm}


%
%   CODE
%

\definecolor{commentColor}{HTML}{adb5bd}
\definecolor{mygray}{rgb}{0.5,0.5,0.5}
\definecolor{stringColor}{HTML}{7048e8}
\definecolor{keywordColor}{HTML}{228be6}
\definecolor{backgroundColor}{HTML}{f1f3f5}
\definecolor{borderColor}{HTML}{f1f3f5}
\definecolor{inlineTextColor}{HTML}{495057}
\definecolor{leftRuleColor}{HTML}{868e96}
\definecolor{numbackgroundColor}{HTML}{f1f3f5}
\definecolor{numColor}{HTML}{adb5bd}

\newtcbox{\inlinebox}{enhanced,nobeforeafter,tcbox raise base,boxrule=0pt,top=0.062em,bottom=0.062em,
  right=0.382em,left=0.382em,arc=0.382em,boxsep=0.1em,before upper={\vphantom{dlg}},
  colframe=white,colback=backgroundColor}

\newcommand{\inlinecode}[1] {
  \inlinebox{\lstinline[language=Python, identifierstyle=\color{inlineTextColor}, basicstyle=\color{inlineTextColor}\ttfamily, keywordstyle=\color{inlineTextColor}]{#1}}
}

%\setmonofont{JetBrainsMono NF}[
%    Contextuals = Alternate,
%    Ligatures = TeX,
%]

\lstset{
  backgroundcolor=\color{backgroundColor},   % choose the background color; you must add \usepackage{color} or \usepackage{xcolor}; should come as last argument
  basicstyle=\small\ttfamily,        % the size of the fonts that are used for the code
  breaklines=true,                 % sets automatic line breaking
  commentstyle=\color{commentColor},    % comment style
  extendedchars=true,              % lets you use non-ASCII characters; for 8-bits encodings only, does not work with UTF-8
  firstnumber=1,                % start line enumeration with line 1000
  frame=single,	                   % adds a frame around the code
  frameshape={RYR}{Y}{Y}{RYR},
  keepspaces=true,                 % keeps spaces in text, useful for keeping indentation of code (possibly needs columns=flexible)
  keywordstyle=\color{keywordColor}\textbf,       % keyword style
  language=Python,                 % the language of the code
  morekeywords={*,...},            % if you want to add more keywords to the set
  numbers=left,                    % where to put the line-numbers; possible values are (none, left, right)
  numbersep=1.5em,                   % how far the line-numbers are from the code
  numberstyle=\tiny\color{commentColor}, % the style that is used for the line-numbers
  rulecolor=\color{borderColor},         % if not set, the frame-color may be changed on line-breaks within not-black text (e.g. comments (green here))
  showstringspaces=false,          % underline spaces within strings only
  showtabs=false,                  % show tabs within strings adding particular underscores
  stepnumber=1,                    % the step between two line-numbers. If it's 1, each line will be numbered
  stringstyle=\color{stringColor},     % string literal style
  tabsize=2,
  frame=l,
  framesep=2.12em,
  framexleftmargin=0em,
  fillcolor=\color{numbackgroundColor},
  rulecolor=\color{leftRuleColor},
  numberstyle=\ttfamily\tiny\color{numColor},
}

\newtcolorbox{codebg}[2][]{%
    colback=gray-1,
    %grow to right by=-10mm,
    %grow to left by=-10mm,
    boxrule=0pt,
    boxsep=0pt,
    breakable,
    enhanced jigsaw,
    rounded corners=east,
    arc=8pt,
    borderline west={1pt}{0pt}{gray-3},
    %title={#2\par},
    %colbacktitle={zitat_bg},
    bottomrule=0pt,
}


\newtcbox{\inlinecodee}{on line, boxrule=0pt, boxsep=0pt, top=2pt, left=2pt, bottom=2pt, right=2pt, colback=gray-2, colframe=white, fontupper={\ttfamily \footnotesize}}

\BeforeBeginEnvironment{minted}{\begin{codebg}}%
\AfterEndEnvironment{minted}{\end{codebg}}%

\BeforeBeginEnvironment{inputminted}{\begin{codebg}}%
\AfterEndEnvironment{inputminted}{\end{codebg}}%

\usepackage{minted}
\setminted{
  autogobble=true,
  breakautoindent=true,
  breaklines=true,
  escapeinside=§§,
  fontfamily=tt,
  fontsize=\footnotesize,
  frame=leftline,
  framerule=0pt,
  framesep=0.2em, % sufficient for up to 4 digits
  numbers=left,
  numbersep=0.2em,
  showspaces=false,
  showtabs=false,
  style=vs, % see: https://pygments.org/styles/
  tabsize=2,
  xleftmargin=1.5em,
  % colors
  bgcolor=gray-1,
}
\usemintedstyle{myown}

\tcbuselibrary{minted}

% minted line numbers
\renewcommand{\theFancyVerbLine}{\sffamily
\textcolor{gray-4}{\scriptsize
\oldstylenums{\arabic{FancyVerbLine}}}}


\begin{document}

\begin{center}
  \Large{Theoretische Elektrotechnik: \textbf{Elektrostatik}}\\
  \small{Elektrische Felder ruhender Ladungen}
\end{center}

\begin{flushright}
  R. Grünert\\
  \today
\end{flushright}

\minisec{Ausgangsbedingungen der Elektrostatik:}
\begin{itemize}
  \item[$\vartriangleright$]$\divv \vec{D} = \rho$
  \item[$\vartriangleright$]$\divv \vec{B} = 0$ (sowieso)
  \item[$\vartriangleright$]$\rot \vec{E} = -\dfrac{\partial \vec{B}}{\partial t} \color{red-8}= 0 $ \\(Statik, das elektrische Feld ist wirbelfrei, da keine Magnetfelder und erst recht keine zeitlich änderlichen Felder betrachtet werden. Das elektrische Feld ist ein reines Quellenfeld)
  \item[$\vartriangleright$] $\rot \vec{B} = \vec{S} + \dfrac{\partial \vec{D}}{\partial t} \color{red-8}=0$\\
  (Statik, keine Stromdichten bzw. keine bewegten Ladungen und erst recht keine zeitliche Änderung dieser)
\end{itemize}

\section{Potentialgleichung}
\subsection{Vorbetrachtung:}
\[\rot \vec{E} = 0\]
Bildung des elektrischen Feldes aus dem Gradienten über eine skalare Potentialfunktion $T$ sollte möglich sein:
\[\vec{E} = -\grad T\]
$\rightarrow$ Das heißt die Rotation des Gradienten des Skalarfeldes (=Vektorfeld) muss 0 sein. Es gibt keine Wirbel im Gradientenfeld.
\[\rot \grad T {\color{red-8}=} 0\]
\[\vecnabla \times (\vecnabla \cdot T) = \underbrace{\vecnabla \times \vecnabla}_{=0}\cdot T\]
(Note: T ist ein skalar, also nur ein Faktor für die Vektoren, daher kann die Reihenfolge getauscht werden)\\
\notebox{Wirbelfreiheit von Quellenfeldern}{
  Quellenfelder (und das \emph{statische} elektrische Feld ist so eines) sind wirbelfrei!\\

  Ein wirbelfreies Feld kann aus dem Gradienten einer skalaren Potentialfunktion abgeleitet werden!
}

\subsection{Herleitung der Potentialgleichung}
\[\divD = \rho\]
Mit der Materialgleichung wird daraus:
\[\divv \epsilon \vec{E} = \rho\]
\[\divE = \frac{\rho}{\epsilon}\]
wobei $\epsilon$ konstant ist (nicht Orts- oder Feldstärkeabhängig).\\
Mit $\vec{E} = -\grad \phi$ gilt:
\[\divv \grad \varphi = -\frac{\rho}{\epsilon}\]
Man kann es auch über den Laplace-Operator ausdrücken:
\[\vecnabla \cdot (\vecnabla \cdot \varphi) = \vecnabla \vecnabla \varphi = -\frac{\rho}{\epsilon}\]
\[\vecnabla^{2}=-\frac{\rho}{\epsilon}\]

\notebox{Die Potentialfunktion}{
  \[\Delta \varphi = -\dfrac{\rho}{\epsilon}\]
}

\subsection{Zwei Sonderfälle der Gleichung}
\minisec{1. Fall: $\rho = 0$:}
Keine \emph{Ladungsdichte} in dem Gebiet $\rightarrow$ Laplacegleichung.
\notebox{Laplace-Gleichung}{
  \[\Delta \varphi = 0\]
}

\minisec{2. Fall: $\rho \ne 0$:}
\notebox{Poisson-Gleichung}{
  \[\Delta \varphi = -\frac{\rho}{\epsilon}\]
}

\section{Lösung der Potentialgleichung für eine Punktquelle}

\begin{figure}[H]
\centering
\resizebox{0.618\textwidth}{!}{\import{graphics/}{potgleichung_pq.pdf_tex}}
\end{figure}

Hier trifft bereits die Laplace-Gleichung zu, da sich außerhalb der Punktquelle keine Ladungsdichten $\rho$ befinden. Geht man vom diskreten Punkt weg, findet man keine weiteren Ladungen. Daher ist die Raumladungsdichte und somit die Potentialgleichung gleich 0.

\[\Delta \, \varphi = 0\]

Am einfachsten lässt sich die Potentialgleichung mithilfe von Kugelkoordinaten lösen. Für die Anwendung des Laplaceoperators in Kugelkoordinaten (und anderen Systemen) gibt es eine Tabelle.

\begin{figure}[H]
\centering
\resizebox{0.618\textwidth}{!}{\import{graphics/}{kugelkoord.pdf_tex}}
\end{figure}

\[\Delta \varphi = \frac{1}{r^{2}} \cdot \frac{\partial}{\partial r} \left(r^{2}\cdot \frac{\partial \varphi}{\partial r} \right) + \frac{1}{r^{2} \cdot \sin\theta}\cdot \frac{\partial}{\partial \theta} \left(\sin\theta \cdot \frac{\partial\varphi}{\partial\theta}\right)+\frac{1}{r^{2}\cdot \sin^{2}\theta}\cdot \frac{\partial^{2}\varphi}{\partial \alpha^{2}}\]

\[\hastobe 0\]

\[\Delta \varphi = \frac{1}{r^{2}} \cdot \frac{\partial}{\partial r} \left(r^{2}\cdot \frac{\partial \varphi}{\partial r} \right) + \color{gray-4}\underbrace{\frac{1}{r^{2} \cdot \sin\theta}\cdot \frac{\partial}{\partial \theta} \left(\sin\theta \cdot \frac{\partial\varphi}{\partial\theta}\right)+\frac{1}{r^{2}\cdot \sin^{2}\theta}\cdot \frac{\partial^{2}\varphi}{\partial \alpha^{2}}}_{\text{entfällt}}\]

Die Anteile/Ableitungen nach den Winkeln $\alpha$ und $\theta$ entfallen aufgrund der Kugelsymmetrie der Anordnung / des Feldes. \textbf{Das Potential $\varphi$ hängt nicht vom Winkel $\alpha$ oder vom Winkel $\theta$ ab. Es hängt nur vom Abstand $r$ vom Ursprung ab!}. Dadurch werden die Ableitungen nach den Winkeln (die ja die Änderungen des Potentials mit der Winkeländerung beschreiben) zu 0, da eine Winkeländerung keine Potentialänderung zu Folge hat.

\begin{align*}
\frac{1}{r^{2}} \cdot \frac{\partial}{\partial r} \cdot (r^{2}\cdot \frac{\partial \varphi}{\partial r}) &= 0 && \text{Multiplikation mit } r^{2}\\
\frac{\partial}{\partial r} \cdot (r^{2}\cdot \frac{\partial \varphi}{\partial r}) &= 0 && \text{1. Integration nach } r\\
r^{2}\cdot \frac{\partial \varphi}{\partial r} &= C_{1}\\
\frac{\partial \varphi}{\partial r} &= \frac{C_{1}}{r^{2}} && \text{2. Integration nach }r\\
\Aboxed{\varphi&=-\frac{C_{1}}{r} + C_{2}}
\end{align*}
Nach dieser Gleichung verläuft das Potential im Abstand r von der Punktladung.\\

\minisec{Bestimmung der Konstanten}

Zur Bestimmung der Integrationskonstanten müssen Raningungen festgelegt werden. Wir wissen, dass, wenn $r$ gegen $\infty$ läuft, dass $\varphi$ im unendlichen dann 0 sein muss. Das ist der phyiskalische Zusammenhang, der uns zur Konstante $C_{2}$ führt. denn $\frac{C_{1}}{r}$ läuft von alleine gegen 0 und damit $\varphi$ insgesamt 0 wird, muss somit $C_{2}$ gleich 0 sein.
\[C_{2} = 0\]

Es bleibt noch $C_{1}$.
\[\varphi = -\frac{C_{1}}{r}\]
aus dem Satz von Gauss:
\[\oiint{\vec{D}\cdot d\vec{A}} = Q = \oiint{\epsilon\vec{E} \cdot d\vec{A}}\]
Mit der zusätzlichen Information, dass es sich hierbei um ein kugelsymmetrisches, also radiales, Feld handelt und damit $\vec{E} = \vec{E}_{r}$ immer senkrecht auf $d\vec{A}$ steht, für ein gegebenes $r$ auf der Hüllfläche konstant und dadurch unabhängig von der Fläche $d\vec{A}$ ist, erhält man
\[Q = \epsilon E_{r} \cdot \oiint{d\vec{A}}\]

Da die Hüllfläche eine Kugelschale ist ergibt sich
\[Q = \epsilon E_{r} \cdot 4 \pi r^{2}\]
\[\boxed{E_{r} = \frac{Q}{4 \pi \epsilon r^{2}}}\]

Damit haben wir aber immer noch nicht die fehlende Konstante oder das Potential ermittelt. Das Potential ergibt sich aus der Feldstärke über den Gradienten
\[\vec{E}_{r} = -\grad \varphi\]
Im Kugelkoordinatensystem bildet man den Gradienten über
\[\vec{E} = - \left( \unitv_{r} \cdot \partialdev{\varphi}{r} + \unitv_{\theta} \cdot \frac{1}{r} \cdot \partialdev{\varphi}{\theta} \cdot \unitv_{\alpha} \cdot \frac{1}{r \sin\theta} \cdot \partialdev{\varphi}{\alpha}\right )\]
\[\vec{E} = - \left( \unitv_{r} \cdot \partialdev{\varphi}{r} + \color{gray-4}\underbrace{\unitv_{\theta} \cdot \frac{1}{r} \cdot \partialdev{\varphi}{\theta} \cdot \unitv_{\alpha} \cdot \frac{1}{r \sin\theta} \cdot \partialdev{\varphi}{\alpha}}_{\text{entfällt}}\right )\]

Wieder entfallen augrund der Kugelsymmetrie bzw. der Unempfindlichkeit des Potentials gegenüber Winkeländerungen die beiden hinteren Terme mit den Ableitungen nach den Winkeln. Es bleibt ein Vektor(feld) mit reiner Radialkomponente.

\[E_{r} = - \partialdev{\varphi}{r}\]
$\varphi$ wird eingesetzt aus der allgemeinen Lösung der Potentialgleichung ($\Delta \varphi = 0$, $C_{2}$ = 0).
\[E_{r} = - \frac{\partial}{\partial r}\left(-\frac{C_{1}}{r}\right)\]
\[E_{r} = -\frac{C_{1}}{r^{2}}\]
Diese Gleichung wird nun mit dem Ergebnis aus dem Gausschen Satz verglichen.
\[C_{1} = -\frac{Q}{4\pi\epsilon}\]
dadurch wird
\[\boxed{\varphi = \frac{Q}{4 \pi \epsilon \cdot r}}\]
Was aus der Schulphysik bekannt ist.

\section{Superpositionsprinzip}

\begin{figure}[H]
\centering
\resizebox{0.618\textwidth}{!}{\import{graphics/}{superpositionsprinzip.pdf_tex}}
\end{figure}

\paragraph{Frage:} Wie groß ist das Potential im Punkt P bei mehreren gegebenen Ladungen? \paragraph{Lösung:} Superposition der Teilbeiträge (Einzelpotentiale im Punkt P)
diskret, N Punktladungen:
\[\varphi_{P} = \sum_{i=1}^{N}{ \frac{Q_{i}}{4\pi \epsilon \cdot r_{i}} }\]

Mit $Q_{i} = \rho_{i} \cdot dV_{i}$

\[\varphi_{P} = \sum_{i=1}^{N}{ \frac{\rho_{i} \cdot dV_{i}}{4\pi \epsilon \cdot r_{i}} }\]

Übergang Summe $\rightarrow$ Integral:
\[\varphi_{P} = \frac{1}{4\pi\epsilon} \iiint{\frac{\overbrace{\rho \cdot dV}^{\text{Punktladungsbeiträge}}}{r}}\]

\notebox{Superposition von Raumladungen}{
\[\varphi_{P} = \frac{1}{4\pi\epsilon} \iiint{\frac{\rho \cdot dV}{r}}\]
}

\notebox{Superposition von Flächenladungen}{
\[\varphi_{P} = \frac{1}{4\pi\epsilon} \iiint{\frac{\sigma \cdot dA}{r}}\]
}

\notebox{Superposition von Linienladungen}{
\[\varphi_{P} = \frac{1}{4\pi\epsilon} \iiint{\frac{\lambda \cdot ds}{r}}\]
}

\section{Spiegelungsmethode}
\paragraph{Prinzip}
\begin{itemize}
  \item[$\vartriangleright$] Leiteroberflächen sind gleichzeitig \emph{Äquipotentialflächen}
  \item[$\vartriangleright$] Das innere eines Leiters ist \emph{feldfrei}
  \item[$\vartriangleright$] Aus einer Leiteroberfläche treten die Feldlinien immer \emph{senkrecht} ein oder aus (siehe Gradient)
  \item[$\vartriangleright$] Die Verteilung einer Ladung ergibt die Flächenladungsdichte \[\sigma  = \frac{dQ}{dA} \rightarrow Q = \iint{\sigma dA}\]
  \item[$\vartriangleright$] Da an der Oberfläche nur die Normalkomponente von $\vec{E}$, $\vec{E}_{n}$, vorhanden ist und $\vec{D} = \epsilon \vec{E}$ gilt, ist auch nur die Normalkomponente von $\vec{D}$, $\vec{D}_{n}$ vorhanden und dadurch ist
  \[Q = \iint{D_{n} \cdot dA}\]
  aus dem Vergleich folgt
  \[D_{n} = \sigma\]
\end{itemize}


Angenommen man hat eine Punktladung, deren Feldlinien auf eine Äquipotentialfläche treffen und gesucht ist das Potential $\varphi_{P}$, das durch die Ladung an einem Punkt $P$ (oder halt in jedem Raumpunkt) verursacht wird.

\begin{figure}[H]
\centering
\resizebox{0.618\textwidth}{!}{\import{graphics/}{spiegel0.pdf_tex}}
\end{figure}

Um diese Problemstellung zu vereinfachen, kann man sich eine entgegengesetzt geladene Ladung $Q'$ vorstellen, die sich im selben Abstand $r$ wie die Ladung $Q$ zur Äquipotentialfläche befindet. Die Ladung $Q'$ vervollständigt das bekannte Feldbild zweier Punktladungen im Abstand $2 r$. Durch die Superposition des Potentials beider Ladungen im Punkt $P$ ergibt sich dann das gesuchte Gesamtpotential $\varphi_{P}$

\begin{figure}[H]
\centering
\resizebox{0.618\textwidth}{!}{\import{graphics/}{spiegel1.pdf_tex}}
\end{figure}

Die Superposition ergibt
\[\varphi_{P} = \frac{1}{4\pi\epsilon} \left( \frac{Q}{r_{1}} {\color{red-8}{-}} \frac{Q}{r_{2}}\right)\]
\[\boxed{\varphi_{P} = \frac{Q}{4\pi\epsilon} \left( \frac{1}{r_{1}} {\color{red-8}{-}} \frac{1}{r_{2}}\right)}\]

\section{Erweiterte Spiegelmethode}
Es stellt sich die Frage, wie man das Potential einer Punktladung bei andersgeformten Äquipotentialflächen, in diesem Fall einer \emph{Kugel}, berechnen kann. Beispielsweise ist eine Punktladung $q$ vor einer leitfähigen Kugel gegeben.

\begin{figure}[H]
\centering
\resizebox{0.618\textwidth}{!}{\import{graphics/}{kugel_punktladung.pdf_tex}}
\end{figure}

Gesucht sind hierbei der Ort der Ersatzladung $q'$, d.h. ihr Abstand vom Mittelpunkt, sowie ihr Betrag.
Kennt man beide Werte, kann man wie bei der einfachen Spiegelmethode beide Felder überlagern, um das Gesamtfeld zu erhalten.
Es ist also die virtuelle Ladung $q'$ gesucht, die das Feld der Ladung $q$ zu der Kugel simuliert.\\

Betrachtet man einen Punkt $P$ auf der Oberfläche der Kugel ergibt sich
\[\phi_{P} = \frac{1}{4\pi\epsilon} \cdot \left( \frac{q}{r_{2}} - \frac{q'}{r_{1}}\right ) \hastobe 0\]
Diese Überlagerung muss null ergeben, denn hier ist das Potential der Kugel zu Null definiert (geerdete Kugel).\\

\begin{figure}[H]
\centering
\resizebox{0.618\textwidth}{!}{\import{graphics/}{kugel_punktladung2.pdf_tex}}
\end{figure}

\subsection{Ort der Spiegelladung}
Zuerst einige Geometriedefinitionen:

\begin{figure}[H]
\centering
\resizebox{0.618\textwidth}{!}{\import{graphics/}{kugel_punktladung3.pdf_tex}}
\end{figure}

Der Mittelpunkt $M$ ist dabei als Ursprung definiert.

Zur Lösung wird der $\emph{Satz des Apollonios}$ angewandt. Dieser bringt hervor, dass das Verhältnis
\[\frac{r_{2}}{r_{1}}\]
\emph{konstant} sein muss, unabhängig davon, wo der Punkt $P$ auf dem Kreis / der Kugel liegt.

\begin{figure}[H]
\centering
\resizebox{0.618\textwidth}{!}{\import{graphics/}{apollonios.pdf_tex}}
\end{figure}
\[b^{2} + c^{2} = 2 \cdot m^{2} + \frac{a^{2}}{2}\]


%\notebox{hey}{\blindtext}
\end{document}
