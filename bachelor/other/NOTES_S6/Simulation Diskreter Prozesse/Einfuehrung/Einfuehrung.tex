\documentclass[11pt, a4paper]{article}

%
%   PACKAGES
%

\usepackage{graphicx}
\usepackage{amsmath}
\usepackage{amssymb}
\usepackage{mathtools}
\usepackage[binary-units=true]{siunitx}
\usepackage[table,xcdraw]{xcolor}
\usepackage{tikz}
\usepackage[most]{tcolorbox}
%\usepackage[english,german]{babel}
\usepackage[english]{babel}
\usepackage{blindtext}
\usepackage{import}
\usepackage{float}
\usepackage{varwidth}
\usepackage{esint}
\usepackage{subfig}
\usepackage[hyphens]{url}
\usepackage{hyperref}
\usepackage{pdfpages}
\usepackage{multicol}
\usepackage{listings}
\usepackage{geometry}
\usepackage{fontspec}
\usepackage[
    backend=biber,
]{biblatex}
\usepackage{todonotes}

%
%   GEOMETRY
%

\geometry{
  a4paper,
  %total={0.618\paperwidth,0.7639\paperheight},
  %left=0.190962\paperwidth,
  total={0.764\paperwidth,0.7639\paperheight},
  left=0.118\paperwidth,
}


%
%   FONT
%

%s\renewcommand{\familydefault}{\sfdefault}

%
%   COLORS
%

% \input{"/home/zamza/Documents/HS/Master/lnic-masters-protocols/preamble/colors.tex"}
\input{"colors.tex"}
\definecolor{yucky}{HTML}{dee2e6}
\definecolor{yuckytext}{HTML}{343a40}

%
%   HIGHLIGHTING
%
\newcommand{\minisec}[1]{\noindent\underline{\textbf{#1}}\\}

%
%   MATH
%

\newcommand\equalhat{\mathrel{\stackon[1.5pt]{=}{\stretchto{%
    \scalerel*[\widthof{=}]{\wedge}{\rule{1ex}{3ex}}}{0.5ex}}}}

\newcommand{\vecnabla}{\vec{\nabla}}
\newcommand{\rot}{\text{rot}\,}
\newcommand{\divv}{\text{div}\,}
\newcommand{\grad}{\text{grad}\,}
\newcommand{\divD}{\divv\vec{D}}
\newcommand{\divB}{\divv\vec{B}}
\newcommand{\divE}{\divv\vec{E}}
\newcommand{\rotD}{\rot\vec{D}}
\newcommand{\rotB}{\rot\vec{B}}
\newcommand{\rotE}{\rot\vec{E}}
\newcommand{\unitv}{\vec{e}}
\newcommand{\partialdev}[2]{\frac{\partial #1}{\partial #2}}

% ! equals
\newcommand{\hastobe}{\stackrel{!}{=}}

% dBm


%
%   HYPERLINKS
%
\hypersetup{
    colorlinks=true,
    linkcolor=blue-6,
    citecolor=blue-8,
    urlcolor=blue-6
}


%
%   FLOWCHART
%

\usetikzlibrary{shapes,arrows}

\tikzstyle{decision} = [diamond, draw, fill=blue!20,
    text width=4.5em, text badly centered, node distance=3cm, inner sep=0pt]
\tikzstyle{block} = [rectangle, draw=yucky, fill=yucky, text=yuckytext,
    text width=10em, text centered, rounded corners, minimum height=4em, minimum width=10em]
\tikzstyle{line} = [draw, -latex']
\tikzstyle{cloud} = [draw, ellipse,fill=red!20, node distance=3cm,
    minimum height=2em]

%
%   NOTE BOX
%

\colorlet{tcb_content_bg}{green-0}
\colorlet{tcb_title_bg}{green-0}
%\tcbset{colback=tcb_content_bg, colbacktitle=tcb_title_bg, colframe=tcb_content_bg, boxrule=0pt, bottomrule=0pt, frame hidden, sharp corners}

\tcbset{
  enhanced,
  sharp corners,
}

\newcommand{\notebox}[2]{
  \vspace{\baselineskip}
  \begin{tcolorbox}[title=#1]{#2}\end{tcolorbox}
  \vspace{\baselineskip}
}

\colorlet{zitat_bg}{gray-0}
\colorlet{zitat_strip}{gray-3}

\newtcolorbox{taskspec}[1][]{%
    colback=gray-1,
    fontupper=\selectfont\ttfamily,
    %grow to right by=-10mm,
    %grow to left by=-10mm,
    boxrule=0pt,
    boxsep=0pt,
    breakable,
    enhanced jigsaw,
    borderline west={1pt}{0pt}{gray-2},
    borderline east={1pt}{0pt}{gray-2},
    borderline north={1pt}{0pt}{gray-2},
    borderline south={1pt}{0pt}{gray-2},
    %colbacktitle={gray-2},
    %coltitle={gray-8},
    %fonttitle={\large\bfseries},
    attach title to upper={},
    #1,
}

\newtcolorbox{zitat}[2][]{%
    colback=zitat_bg,
    grow to right by=-10mm,
    grow to left by=-10mm,
    boxrule=0pt,
    boxsep=0pt,
    breakable,
    enhanced jigsaw,
    borderline west={4pt}{0pt}{zitat_strip},
    title={#2\par},
    colbacktitle={zitat_bg},
    coltitle={gray-8},
    fonttitle={\large\bfseries},
    attach title to upper={},
    #1,
}

\newcommand{\notebo}[2]{
  \vspace{\baselineskip}
  \begin{tcolorbox}[enhanced,
  sharp corners,
  boxrule=0pt,
  toptitle=0.1cm+1pt,%
  bottomtitle=-0.1cm+0.5em,%
  colframe=red-0,colback=red-0,coltitle=red-7,
  title style=red-0,
  fonttitle=\bfseries,fontupper=\normalsize,title=#1]{#2}\end{tcolorbox}
  \vspace{\baselineskip}
}

\newtcbtheorem[number within=chapter]{thm}{Theorem}{
  theorem style=change apart,
  enhanced,
  frame hidden,interior hidden,
  sharp corners,
  boxrule=0pt,
  left=0.2cm,right=0.2cm,top=0.2cm,
  toptitle=0.1cm+1pt,%        <-- I used your values here
  bottomtitle=-0.1cm+0.5em,%  <-- I used your values here
  colframe=white!25!black,colback=white,coltitle=white,
  title style=white!25!black,
  bottomrule=1pt,%  <-- reserve space
  borderline south={1pt}{0pt}{white!25!black},%---- draw line
  fonttitle=\bfseries,fontupper=\normalsize}{thm}


%
%   CODE
%

\definecolor{commentColor}{HTML}{adb5bd}
\definecolor{mygray}{rgb}{0.5,0.5,0.5}
\definecolor{stringColor}{HTML}{7048e8}
\definecolor{keywordColor}{HTML}{228be6}
\definecolor{backgroundColor}{HTML}{f1f3f5}
\definecolor{borderColor}{HTML}{f1f3f5}
\definecolor{inlineTextColor}{HTML}{495057}
\definecolor{leftRuleColor}{HTML}{868e96}
\definecolor{numbackgroundColor}{HTML}{f1f3f5}
\definecolor{numColor}{HTML}{adb5bd}

\newtcbox{\inlinebox}{enhanced,nobeforeafter,tcbox raise base,boxrule=0pt,top=0.062em,bottom=0.062em,
  right=0.382em,left=0.382em,arc=0.382em,boxsep=0.1em,before upper={\vphantom{dlg}},
  colframe=white,colback=backgroundColor}

\newcommand{\inlinecode}[1] {
  \inlinebox{\lstinline[language=Python, identifierstyle=\color{inlineTextColor}, basicstyle=\color{inlineTextColor}\ttfamily, keywordstyle=\color{inlineTextColor}]{#1}}
}

%\setmonofont{JetBrainsMono NF}[
%    Contextuals = Alternate,
%    Ligatures = TeX,
%]

\lstset{
  backgroundcolor=\color{backgroundColor},   % choose the background color; you must add \usepackage{color} or \usepackage{xcolor}; should come as last argument
  basicstyle=\small\ttfamily,        % the size of the fonts that are used for the code
  breaklines=true,                 % sets automatic line breaking
  commentstyle=\color{commentColor},    % comment style
  extendedchars=true,              % lets you use non-ASCII characters; for 8-bits encodings only, does not work with UTF-8
  firstnumber=1,                % start line enumeration with line 1000
  frame=single,	                   % adds a frame around the code
  frameshape={RYR}{Y}{Y}{RYR},
  keepspaces=true,                 % keeps spaces in text, useful for keeping indentation of code (possibly needs columns=flexible)
  keywordstyle=\color{keywordColor}\textbf,       % keyword style
  language=Python,                 % the language of the code
  morekeywords={*,...},            % if you want to add more keywords to the set
  numbers=left,                    % where to put the line-numbers; possible values are (none, left, right)
  numbersep=1.5em,                   % how far the line-numbers are from the code
  numberstyle=\tiny\color{commentColor}, % the style that is used for the line-numbers
  rulecolor=\color{borderColor},         % if not set, the frame-color may be changed on line-breaks within not-black text (e.g. comments (green here))
  showstringspaces=false,          % underline spaces within strings only
  showtabs=false,                  % show tabs within strings adding particular underscores
  stepnumber=1,                    % the step between two line-numbers. If it's 1, each line will be numbered
  stringstyle=\color{stringColor},     % string literal style
  tabsize=2,
  frame=l,
  framesep=2.12em,
  framexleftmargin=0em,
  fillcolor=\color{numbackgroundColor},
  rulecolor=\color{leftRuleColor},
  numberstyle=\ttfamily\tiny\color{numColor},
}

\newtcolorbox{codebg}[2][]{%
    colback=gray-1,
    %grow to right by=-10mm,
    %grow to left by=-10mm,
    boxrule=0pt,
    boxsep=0pt,
    breakable,
    enhanced jigsaw,
    rounded corners=east,
    arc=8pt,
    borderline west={1pt}{0pt}{gray-3},
    %title={#2\par},
    %colbacktitle={zitat_bg},
    bottomrule=0pt,
}


\newtcbox{\inlinecodee}{on line, boxrule=0pt, boxsep=0pt, top=2pt, left=2pt, bottom=2pt, right=2pt, colback=gray-2, colframe=white, fontupper={\ttfamily \footnotesize}}

\BeforeBeginEnvironment{minted}{\begin{codebg}}%
\AfterEndEnvironment{minted}{\end{codebg}}%

\BeforeBeginEnvironment{inputminted}{\begin{codebg}}%
\AfterEndEnvironment{inputminted}{\end{codebg}}%

\usepackage{minted}
\setminted{
  autogobble=true,
  breakautoindent=true,
  breaklines=true,
  escapeinside=§§,
  fontfamily=tt,
  fontsize=\footnotesize,
  frame=leftline,
  framerule=0pt,
  framesep=0.2em, % sufficient for up to 4 digits
  numbers=left,
  numbersep=0.2em,
  showspaces=false,
  showtabs=false,
  style=vs, % see: https://pygments.org/styles/
  tabsize=2,
  xleftmargin=1.5em,
  % colors
  bgcolor=gray-1,
}
\usemintedstyle{myown}

\tcbuselibrary{minted}

% minted line numbers
\renewcommand{\theFancyVerbLine}{\sffamily
\textcolor{gray-4}{\scriptsize
\oldstylenums{\arabic{FancyVerbLine}}}}


\begin{document}

\begin{center}
  \Large{Simulation Diskreter Prozesse: Einführung}
\end{center}

\begin{flushright}
  R. Grünert\\
  \today
\end{flushright}

\section{Einführung}
\begin{zitat}{Simulationen}
  Alle Modelle sind falsch aber manche sind nützlich.
\end{zitat}

Für Simulationen eignen sich \emph{analoge} oder auch \emph{digitale/diskrete} Modelle. Analoge Modelle sind reale/phsyische Modelle, wie Miniaturen. Bspw. Flugzeugmodelle im Windkanal.
Digitale Modelle nutzen rechnergestützte Verfahren, wie z.B. die \emph{FEM}. Bei der FEM werden Geometrie und Randbeingungen am Computer vorgegeben. Die Geometrie wird in Grundblöcke, \glqq Finite Elemente\grqq, eingeteilt, für welche dann jeweils bestimmte Gleichungen, z.B. die Maxwellschen Gleichungen, ausgewertet werden.\\

Häufig auftretende Probleme sind sogenannte \glqq Queuing Probleme \grqq, also Warteschlangenprobleme, die durch die Diskrete Simulation gelöst werden können (z.B. mithilfe von MATLAB Simulink). In diese Kategorie fallen z.B. die Übertragung von Datenpaketen, Flugzverkehr, Transport, Verkehrskreuzungen, Arztwarteschlangen ($\rightarrow$ Markov-Prozesse).\\

Weiterhin bedient man sich oft dem Zufall und wendet stochastische Simulationsmethoden, wie z. B. \glqq Monte Carlo \grqq Methoden an.

\notebox{Simulation}{
\begin{description}
  \item[Simulation:] Verfahren, bei dem für ein reales \emph{oder} imaginäres System ein Modell erstellt wird, das experimentell untersucht werden kann.\\
\end{description}
}

\emph{Ziel} von Simulationen ist es, neue Erkenntnisse über ein System zu gewinnen und aus diesen Erkenntnissen Handlungsanweisungen, also reale Maßnahmen oder technische Umsetzungen, abzuleiten.

\notebox{Simulation}{
\begin{description}
\item[Diskrete Simulation:] Ein System wird in einem Computermodell in Form von Software abgebildet.
\item[Stochastische Simulation:] Nachbildung eines Systems ($rightarrow$ Zufall) auf einen Computer zwecks Untersuchung der Eigenschaften dieses Systems, z.B. \grqq Monte-Carlo-Methode\glqq.
\end{description}
}


\begin{figure}[H]
\centering
\resizebox{0.75\textwidth}{!}{\import{graphics/}{stochastik.pdf_tex}}
\end{figure}

\notebo{Simulation}{
  Eine Simulation ist nicht das gleiche wie ein Experiment oder eine mathematisch-analytische Methode.
}

\begin{description}
\item[Gemeinsamkeit mit empirischem Experiment:]
        Der empirische Ansatz (d.h. das Zählen, Mesen, usw.)
  \item[Gemeinsamkeit mit mathematischer Untersuchung:]
        Ein mathematisches Modell (= (fehlerhaftes) Abbild der Wirklichkeit)
\end{description}

\noindent\textbf{Vorteile}
\begin{description}
  \item[ggü. Experiment:]
        \begin{itemize}
        \item z.B. Simulationszeit 1 Jahr $\rightarrow$ Rechenzeit 1ms
        \item man spart sich die Kosten für den Versuchsaufbau
        \end{itemize}
  \item[ggü. math. Methode:]
        analytisch nicht lösbare Probleme werden numerisch lösbar
\end{description}

\section{Modell und Wirklichkeit}
\subsection{Geozentrisches Weltbild, Aristoteles}
Aristoteles begann mit dem Geozentrischen Weltbild. Die Erde steht im Zentrum, um sie herum sind konzentrische Bahnen von Planeten/Sternen, etc.

\begin{figure}[H]
\centering
\resizebox{0.618\textwidth}{!}{\import{graphics/}{aristot.pdf_tex}}
\end{figure}

Man sollte sich immer die Fragen stellen: \glqq Welches Ziel hat mein Modell?\grqq und \glqq Reicht mein Modell für seine Anwendungen?\grqq

\subsection{Epizyklen, Ptolemaeus}
Das aristotelische Modell machte einige Vorhersagen nicht präzise und war somit ungeeignet für Berechnungen. Ptolemaeus erfand sein Modell, das immer noch geozentrisch war, jedoch trotzdem exakte Vorhersagen lieferte.

\begin{figure}[H]
\centering
\resizebox{0.618\textwidth}{!}{\import{graphics/}{ptoli.pdf_tex}}
\end{figure}

\subsection{Heliozentrik, Kopernikus}
Kopernikus brachte ein realitäts\emph{näheres} Modell, die Sonne steht im Zentrum und alles andere fliegt in Kreisen darum. Allerdings sind die Vorhersagen bei diesem Modell noch schlechter als beim ptolemaeischen.

\begin{figure}[H]
\centering
\resizebox{0.618\textwidth}{!}{\import{graphics/}{kopi.pdf_tex}}
\end{figure}

\subsection{Gallileo, Einstein, usw...}
\notebo{Modell und Wirklichkeit}{
  \textbf{Es ist egal, ob das Modell der Realität exakt entspricht. Wichtig ist, dass es \glqq richtige\grqq        Vorhersagen trifft.}\\
  {\small Realitätsnähe ist je nach Anwendung möglicherweise ungewollt, siehe die verschiedenen Modelle von Licht (Teilchenmodell, Wellenmodell, Maxwellsche oder Einsteinsche Gleichungen (u.U. overkill))}
}

\notebo{Modell und Wirklichkeit}{
  Modelle sind nicht die Wirklichkeit. Es gibt kein exaktes Modell der Wirklichkeit (denn sonst bräuchten wir die Wirklichkeit nicht).
}

Ziel ist, die richtige Stufe von Modellen zu finden.
\glqq Nur so gut wie nötig\grqq.


%\notebox{hey}{\blindtext}
\end{document}
