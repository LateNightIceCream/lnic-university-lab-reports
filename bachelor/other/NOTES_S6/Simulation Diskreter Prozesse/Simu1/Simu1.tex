\documentclass[11pt, a4paper]{article}

%%% SST LAB PROTOCOLL PREAMBLE
%%% 2019
%%%%%%%%%%%%%%%%%%%%%%%%%%%%%%%


%%% PACKAGES
%%%%%%%%%%%%%%%%%%%%%%%%%%%

\usepackage[ngerman]{babel}

\usepackage[utf8]{inputenc}
\usepackage{amsmath}
\usepackage{pgfplots}
\usepackage{tikz}
\usepackage[many]{tcolorbox}
\usepackage{graphicx}
\graphicspath{ {./graphics/} }
\usepackage{pdfpages}
\usepackage{dashrule}
\usepackage{float}
\usepackage{siunitx}
\usepackage{trfsigns}
\usepackage{booktabs}
\usepackage[european]{circuitikz}
\usepackage{tcolorbox}

%%% DOCUMENT GEOMETRY
%%%%%%%%%%%%%%%%%%%%%%%%%%%

\usepackage{geometry}
\geometry{
 a4paper,
 total={0.6180339887498948\paperwidth,0.6180339887498948\paperheight},
 top = 0.1458980337503154\paperheight,
 bottom = 0.1458980337503154\paperheight
 }
\setlength{\jot}{0.013155617496424828\paperheight}
\linespread{1.1458980337503154}

\setlength{\parskip}{0.013155617496424828\paperheight} % paragraph spacing


%%% COLORS
%%%%%%%%%%%%%%%%%%%%%%%%%%%

\definecolor{red1}{HTML}{f38181}
\definecolor{yellow1}{HTML}{fce38a}
\definecolor{green1}{HTML}{95e1d3}
\definecolor{blue1}{HTML}{66bfbf}
\definecolor{hsblue}{HTML}{00b1db}
\definecolor{hsgrey}{HTML}{afafaf}

%%% CONSTANTS
%%%%%%%%%%%%%%%%%%%%%%%%%%%
\newlength{\smallvert}
\setlength{\smallvert}{0.0131556\paperheight}


%%% COMMANDS
%%%%%%%%%%%%%%%%%%%%%%%%%%%

% differential d
\newcommand*\dif{\mathop{}\!\mathrm{d}}

% horizontal line
\newcommand{\holine}[1]{
  	\begin{center}
	  	\noindent{\color{hsgrey}\hdashrule[0ex]{#1}{1pt}{3mm}}\\%[0.0131556\paperheight]
  	\end{center}
}

% mini section
\newcommand{\minisec}[1]{ \noindent\underline{\textit {#1} } \\}

% quick function plot
\newcommand{\plotfun}[3]{
  \vspace{0.021286\paperheight}
  \begin{center}
    \begin{tikzpicture}
      \begin{axis}[
        axis x line=center,
        axis y line=center,
        ]
        \addplot[draw=red1][domain=#2:#3]{#1};
      \end{axis}
    \end{tikzpicture}
  \end{center}
}

% box for notes
\newcommand{\notebox}[1]{

\tcbset{colback=white,colframe=green1!100!black,title=Note!,width=0.618\paperwidth,arc=0pt}

 \begin{center}
  \begin{tcolorbox}[]
   #1 
  \end{tcolorbox}
 
 \end{center} 
 
}

% box for equation
\newcommand{\eqbox}[2]{
	
	\tcbset{colback=white,colframe=green1!100!black,title=,width=#2,arc=0pt}
	
	\begin{center}
		\begin{tcolorbox}[ams align*]
				#1
		\end{tcolorbox}
		
	\end{center} 
	
}
% END OF PREAMBLE

\begin{document}

\begin{center}
  \Large{Simulation diskreter Prozesse: Simulation(?)}
\end{center}

\begin{flushright}
  R. Grünert\\
  \today
\end{flushright}

\section{Simulation}
\noindent Zufallsvariablen $Y$\\
\noindent Funktion $h$\\
\noindent Inputs $\vec{x} = (x_{1}, ..., x_{n})\rightarrow$ Zufallsvektor, Verteilung von $x$ ist meist bekannt\\

\begin{figure}[H]
\centering
\resizebox{\textwidth}{!}{\import{graphics/}{flowy.pdf_tex}}
\end{figure}

Aus $Y$ (Output) lassen sich dann statistische Kenngrößen, wie Mittelwert, Standardabweichung, etc. ermitteln.

\section{Zufallszahlen mit Matlab}
Mit Matlab soll das Gesetz der großen Zahlen überprüft werden. Es wird zum Vergleich eine Normalverteilung über die Funktion
\[F(x) = \]
generiert und grafisch ausgegeben.\\

Danach wird eine Gleichverteilung mit dem \emph{rand}-Befehl erzeugt und in einem Histogramm dargestellt. Diese Gleichverteilung ist nur gut für eine große Anzahl an zufälligen Werten.\\

Nun werden mehrere Gleichverteilungen überlagert mithilfe eines \emph{Averaging-Verfahrens} über eine einfache for-Schleife.\\


Beispiel: Viele gleichverteilte Rauschprozesse überlagern sich $\rightarrow$ Normalverteilung.
ROT: Bei mehr als 30 Überlagerungen kann man Tests auf Normalverteilungen anwenden.

\section{}
\section{Zustand eines diskreten dynamischen Systems}

Zustandsänderungen nur zu diskreten Zeitpunkten möglich (innerhalb eines vorgegebenen Zeitrasters).

BILD EINFÜGEN

\begin{description}
        \item[Zustand:] Ist die Menge der pronzipiell möglichen Zustände kontinueierlich, hat das System einen \textbf{kontinuierlichen Zustandsraum}. Sonst: \textbf{Diskreter oder endlicher Zustandsraum}.
\end{description}

\section{Ereignisse und Aktivitäten}
$\rightarrow$ (sprunghafte) Änderung eines Systems. $\rightarrow$ Änderung ist ein Ereignis (event).
Der Systemzustand ändert sich nur, wenn ein bestimmtes Ereignis (Zustandsänderung) stattfindet.

\begin{description}
\item[Ereignis:] ist ein Geschehen, das keine \emph{Realzeit} (Simulationszeit) in Anspruch nimmt (in der betrachteten Zeitebene), aber \emph{Rechenzeit} beansprucht!
\item[Simulationszeit:] Die bei der Simulation eines Modells im Rechner durch die Software abgebildete Realzeit.
\end{description}

Ein Ereignis tritt in einem Zeitpunkt ein. Diesem Zeitpunkt kann man das Ereignis zuordnen $\rightarrow$ \emph{Zeitstempel} des Ereignisses.\\

Bei der Simulation eines Modells im Rechner durch die Software abgebildete Realzeit wird als Simulationszeit bezeichnet.

\begin{description}
\item[Aktivitiät (activity):] Ein Vorgang, der zwischen einem Anfangsereignis und einem später folgenden Endereignis abläuft.
\end{description}

Die Aktivität beansprucht \textbf{Realzeit} ()da Zeitdauer vergeht).
Sie ändert den Zustand eines Systems nicht.

\begin{figure}[H]
\centering
\resizebox{\textwidth}{!}{\import{graphics/}{activity.pdf_tex}}
\end{figure}

\begin{description}
\item[Rechenzeit:] Zeitaufwand für die Ausführung eines Simulationsprogramms.
  \item[Ereignisse] benötigen \emph{Rechenzeit} (Zustand wird geändert, Neuberechnung des Systemzustands).
  \item[Aktivität:] Keine Änderung des Zustandes (keine Rechenleistung erforderlich, Zeiten werden nur zugeordnet).
\end{description}

\begin{table}[H]
  \centering
  \begin{tabular}{c|c|c}
    & Realzeit==Simulationszeit & Rechenzeit\\
    \hline
    Ereignisse & Nein & Ja\\
    Aktivitäten & Ja & Nein
  \end{tabular}
\end{table}

\textbf{Aktivitäten} sind \emph{deterministisch}, falls ihre Dauern \emph{vorgegeben} sind. Sie sind \emph{stochastisch}, wenn das Ender der Aktivität vom \emph{Zufall} abhängt.

\begin{description}
  \item[Fahrt Auto von A nach B:] stochastische Aktivität
        (abhängig von Wetter, Verkehrsdichte(t), ...)
\end{description}

\begin{description}
  \item[Transport Werkstück auf Fließband mit $v=$const.:] deterministische Aktivität
\end{description}

\section{Nebenläufigkeit von Aktivitäten}
\emph{Aktivitäten können zumindest teilweise gleichzeitig stattfinden} $\rightarrow$ \textbf{parallele Aktivitäten}.

Besteht kein kausaler Zusammenhang, d.h. die Aktivitäten beeinflussen sich nicht gegenseitig, spricht man von \textbf{nebenläufigen Aktivitäten} (können parallel sein (immer nebenläufig) oder sequentiell).

\begin{figure}[H]
\centering
\resizebox{0.382\textwidth}{!}{\import{graphics/}{paraleli.pdf_tex}}
\end{figure}

\section{Abhängige und unabhängige Ereignisse}
\minisec{Kausale Abhängigkeit}

\begin{figure}[H]
\centering
\resizebox{0.618\textwidth}{!}{\import{graphics/}{abhgkg.pdf_tex}}
\end{figure}

\begin{description}
\item[Abhängiges Ereignis (conditional event):] Wenn  sein Eintrittszeitpunkt vom Eintreffen eines anderen Ereignisses abhängt (das im gleichen Zeitpunkt stattfindet). Sonst: unabhängiges Ereignis (unconditional event).
\end{description}

\section{Prozesse}
\begin{description}
\item[Prozesse:] Ein Prozess ist ein \textbf{dynamisches System}, das mit einer \textbf{Ablauflogik} ausgestattet ist. Die Ablauflogik bestimmt die Menge der möglichen Verläufe der Prozessinstanzen. Spielt dabei der Zufall eine Rolle $\rightarrow$ stochastischer Prozess.
\end{description}

Kreuzung: Ablauflogik: Rechts vor Links.

\begin{figure}[H]
\centering
\resizebox{\textwidth}{!}{\import{graphics/}{kreuz.pdf_tex}}
\end{figure}



%\notebox{hey}{\blindtext}
\end{document}
