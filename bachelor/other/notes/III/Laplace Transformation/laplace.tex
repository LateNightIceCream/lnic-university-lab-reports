\documentclass[a4paper, 12pt]{article}

\setlength\parindent{0pt}

\usepackage{amsmath}
\usepackage{pgfplots}
\usepackage{tikz}

\newcommand{\holine}{
  \noindent\rule{\textwidth}{0.618pt}\\[0.021286\paperheight]
}

\newcommand{\minisec}[1]{ \underline{\textit {#1} } \\[0.021286\paperheight]}

\newcommand{\plotfun}[3]{
  \vspace{0.021286\paperheight}
  \begin{center}
    \begin{tikzpicture}
      \definecolor{plotcol}{HTML}{3c3d47}
      \begin{axis}[
        axis x line=center,
        axis y line=center,
        ]
        \addplot[draw=plotcol][domain=#2:#3]{#1};
      \end{axis}
    \end{tikzpicture}
  \end{center}
}


\begin{document}

\section*{Laplace Transformation}


\subsection*{Ausgangslage}
Die Fourieranalyse stellt die Konvergenzbedingung:

\[ \int_{-\infty}^{\infty}{\mid{x(t)\mid} \cdot dt} < \infty\]

Das bedeutet, dass Signale im undendlichen konvergieren / beschränkt sein
müssen, um eine Fouriertransformation auf ihnen anwenden zu können. Beispiele
sind $\cos{t}$, $e^{-t}$, $\delta({t})$.\\

Sind die zu analysierenden Signale jedoch unbeschränkt/(über alle Grenzen
wachsend), ist die Konvergenzbedingung der FT nicht mehr gegeben. Dies gilt z.B. für $e^t$, $1(t)$\\

\plotfun{exp(x)}{-2}{2}

Eine Rechenhilfe bietet die \emph{Laplace Transformation}. Eingeführt wird die
\emph{komplexe Frequenz}:

\[ j \omega \rightarrow p = \sigma + j\omega \]

\pagebreak
\holine
\minisec{Hinweis}
Es gilt die Darstellung des Cosinus mithilfe zweier, entgegengesetzt
rotierender, komplexer Zeiger:
\[ x(t) = X_0 \cdot \cos{\omega_0 t} = \frac{X_0}{2}\left( e^{j\omega t} + e^{-j
      \omega t}\right) \]

Wird außer der Rotation der Drehzeiger noch eine \emph{zeitliche Änderung der
  Zeigerlänge} (Amplitude) zugelassen, so folgt für die komplexe Frequenz:

\[ p = \sigma + j \omega \]

\holine

Die LT erfasst somit eine wesentlich größere Klasse von Zeitfunktionen /
Signalen und löst somit das Ausgangsproblem

\subsection*{Mathematisches Konzept}

Die Laplacetransformation bildet, wie die Fouriertransformation, ein Bindeglied
zwischen Zeit und Frequenzbereich, mit dem Unterschied, dass hier die komplexe
Frequenz p verwendet wird.

\begin{align*}
  \text{Eingang:  } x(t) \leftrightarrow X(p) = \text{LT}\{x(t)\}\\
  \text{Übertragung:  } g(t) \leftrightarrow G(p) = \text{LT}\{g(t)\}\\
  \text{Ausgang:  } y(t) \leftrightarrow Y(p) = \text{LT}\{y(t)\}\\
\end{align*}

\minisec{Forderung}
\[ \lim_{t\rightarrow \infty}{ x(t) } = 0\]
Bedeutet also, dass die Funktion x(t) abklingen muss!
Dies ist jedoch bei vielen technischen Systemen (sin-, cos-, Sprung-,
Rampenfunktion) nicht erfüllt und führt zu \emph{Konvergenzschwierigkeiten}.

\minisec{Ausweg}
anstatt $x(t) \rightarrow x(t) \cdot e^{-\sigma t}$, wobei $\sigma > 0$\\
Die Forderung wird dann durch geeignete Wahl von $\sigma$ herstellbar!\\

\\$\Rightarrow e^{-\sigma t}$ ist ein \emph{konvergenzerzwingender Faktor} für
rationale Funktionen.\\

Aus dem Fourierintegral:

\[ \int_{-\infty}^{+\infty}{ x(t) \cdot e^{-j\omega t} dt }\]
wird dann mit $x(t) = x(t) \cdot e^{-\sigma t}$:

\[ \int_{-\infty}^{+\infty}{ x(t) \cdot e^{-\sigma t} \cdot e^{-j\omega t} dt }\
  = \int_{-\infty}^{+\infty}{ x(t) \cdot e^{-p\cdot t}dt }\]


Dies entspricht der Fouriertransformation einer \emph{exponentiell gedämpften
  Zeitfunktion}! Es lässt sich außerdem erkennen, dass für den Fall $\sigma = 0$ das Integral zum
regulären Fourierintegral wird.

\holine
Das Integral der Laplacetransformation, bei der die untere Grenze $-\infty$ ist,
bezeichnet man als \emph{bilaterale Laplacetransformation}. Ist diese Grenze
gleich 0, spricht man von der \emph{unilateralen Laplacetransformation}, bei
welcher die Anfangsbedingungen berücksichtigt werden.\\
Für Signale, deren Funktionswerte zu Zeiten $t < 0$ gleich 0 sind, ist die unilaterale gleich der
bilateralen Laplacetransformation. Dies gilt insbesondere für \emph{kausale Systeme}.

\holine
Die Konvergenz des Laplace\emph{integrals} muss gegeben sein. Diese ist vom Wert
von $p$ abhängig, es muss also
herausgefunden werden, für welche werte von $p$ das Integral nicht
unendlich groß wird sondern konvergiert.\\
Diese Analyse führt zum {\bf{Konvergenzbereich}} der Laplacetransformation/des Integrals.\\


Real- und Imaginärteil von $p$, also $\sigma$ und $\omega$, lassen sich in einer
komplexen Ebene darstellen.

\end{document}