\documentclass[a4paper, 12pt]{article}

\usepackage{titlesec}
\titleformat{\section}{
    \usefont{T1}{qhv}{b}{n}\selectfont} % "qhv" - TeX Gyre Heros, "b" - bold
    {} 
    {0em}
    {\Huge}

\titleformat{\subsection}{
    \usefont{T1}{qhv}{b}{n}\selectfont} % "qhv" - TeX Gyre Heros, "b" - bold
    {} 
    {0em}
    {\large}

 \titleformat{\subsubsection}{
    \usefont{T1}{qhv}{b}{n}\selectfont} % "qhv" - TeX Gyre Heros, "b" - bold
    {} 
    {0em}
    {\normalsize}

%
%   PACKAGES
%

\usepackage{graphicx}
\usepackage{amsmath}
\usepackage{amssymb}
\usepackage{mathtools}
\usepackage[binary-units=true]{siunitx}
\usepackage[table,xcdraw]{xcolor}
\usepackage{tikz}
\usepackage[most]{tcolorbox}
%\usepackage[english,german]{babel}
\usepackage[english]{babel}
\usepackage{blindtext}
\usepackage{import}
\usepackage{float}
\usepackage{varwidth}
\usepackage{esint}
\usepackage{subfig}
\usepackage[hyphens]{url}
\usepackage{hyperref}
\usepackage{pdfpages}
\usepackage{multicol}
\usepackage{listings}
\usepackage{geometry}
\usepackage{fontspec}
\usepackage[
    backend=biber,
]{biblatex}
\usepackage{todonotes}

%
%   GEOMETRY
%

\geometry{
  a4paper,
  %total={0.618\paperwidth,0.7639\paperheight},
  %left=0.190962\paperwidth,
  total={0.764\paperwidth,0.7639\paperheight},
  left=0.118\paperwidth,
}


%
%   FONT
%

%s\renewcommand{\familydefault}{\sfdefault}

%
%   COLORS
%

% \input{"/home/zamza/Documents/HS/Master/lnic-masters-protocols/preamble/colors.tex"}
\input{"colors.tex"}
\definecolor{yucky}{HTML}{dee2e6}
\definecolor{yuckytext}{HTML}{343a40}

%
%   HIGHLIGHTING
%
\newcommand{\minisec}[1]{\noindent\underline{\textbf{#1}}\\}

%
%   MATH
%

\newcommand\equalhat{\mathrel{\stackon[1.5pt]{=}{\stretchto{%
    \scalerel*[\widthof{=}]{\wedge}{\rule{1ex}{3ex}}}{0.5ex}}}}

\newcommand{\vecnabla}{\vec{\nabla}}
\newcommand{\rot}{\text{rot}\,}
\newcommand{\divv}{\text{div}\,}
\newcommand{\grad}{\text{grad}\,}
\newcommand{\divD}{\divv\vec{D}}
\newcommand{\divB}{\divv\vec{B}}
\newcommand{\divE}{\divv\vec{E}}
\newcommand{\rotD}{\rot\vec{D}}
\newcommand{\rotB}{\rot\vec{B}}
\newcommand{\rotE}{\rot\vec{E}}
\newcommand{\unitv}{\vec{e}}
\newcommand{\partialdev}[2]{\frac{\partial #1}{\partial #2}}

% ! equals
\newcommand{\hastobe}{\stackrel{!}{=}}

% dBm


%
%   HYPERLINKS
%
\hypersetup{
    colorlinks=true,
    linkcolor=blue-6,
    citecolor=blue-8,
    urlcolor=blue-6
}


%
%   FLOWCHART
%

\usetikzlibrary{shapes,arrows}

\tikzstyle{decision} = [diamond, draw, fill=blue!20,
    text width=4.5em, text badly centered, node distance=3cm, inner sep=0pt]
\tikzstyle{block} = [rectangle, draw=yucky, fill=yucky, text=yuckytext,
    text width=10em, text centered, rounded corners, minimum height=4em, minimum width=10em]
\tikzstyle{line} = [draw, -latex']
\tikzstyle{cloud} = [draw, ellipse,fill=red!20, node distance=3cm,
    minimum height=2em]

%
%   NOTE BOX
%

\colorlet{tcb_content_bg}{green-0}
\colorlet{tcb_title_bg}{green-0}
%\tcbset{colback=tcb_content_bg, colbacktitle=tcb_title_bg, colframe=tcb_content_bg, boxrule=0pt, bottomrule=0pt, frame hidden, sharp corners}

\tcbset{
  enhanced,
  sharp corners,
}

\newcommand{\notebox}[2]{
  \vspace{\baselineskip}
  \begin{tcolorbox}[title=#1]{#2}\end{tcolorbox}
  \vspace{\baselineskip}
}

\colorlet{zitat_bg}{gray-0}
\colorlet{zitat_strip}{gray-3}

\newtcolorbox{taskspec}[1][]{%
    colback=gray-1,
    fontupper=\selectfont\ttfamily,
    %grow to right by=-10mm,
    %grow to left by=-10mm,
    boxrule=0pt,
    boxsep=0pt,
    breakable,
    enhanced jigsaw,
    borderline west={1pt}{0pt}{gray-2},
    borderline east={1pt}{0pt}{gray-2},
    borderline north={1pt}{0pt}{gray-2},
    borderline south={1pt}{0pt}{gray-2},
    %colbacktitle={gray-2},
    %coltitle={gray-8},
    %fonttitle={\large\bfseries},
    attach title to upper={},
    #1,
}

\newtcolorbox{zitat}[2][]{%
    colback=zitat_bg,
    grow to right by=-10mm,
    grow to left by=-10mm,
    boxrule=0pt,
    boxsep=0pt,
    breakable,
    enhanced jigsaw,
    borderline west={4pt}{0pt}{zitat_strip},
    title={#2\par},
    colbacktitle={zitat_bg},
    coltitle={gray-8},
    fonttitle={\large\bfseries},
    attach title to upper={},
    #1,
}

\newcommand{\notebo}[2]{
  \vspace{\baselineskip}
  \begin{tcolorbox}[enhanced,
  sharp corners,
  boxrule=0pt,
  toptitle=0.1cm+1pt,%
  bottomtitle=-0.1cm+0.5em,%
  colframe=red-0,colback=red-0,coltitle=red-7,
  title style=red-0,
  fonttitle=\bfseries,fontupper=\normalsize,title=#1]{#2}\end{tcolorbox}
  \vspace{\baselineskip}
}

\newtcbtheorem[number within=chapter]{thm}{Theorem}{
  theorem style=change apart,
  enhanced,
  frame hidden,interior hidden,
  sharp corners,
  boxrule=0pt,
  left=0.2cm,right=0.2cm,top=0.2cm,
  toptitle=0.1cm+1pt,%        <-- I used your values here
  bottomtitle=-0.1cm+0.5em,%  <-- I used your values here
  colframe=white!25!black,colback=white,coltitle=white,
  title style=white!25!black,
  bottomrule=1pt,%  <-- reserve space
  borderline south={1pt}{0pt}{white!25!black},%---- draw line
  fonttitle=\bfseries,fontupper=\normalsize}{thm}


%
%   CODE
%

\definecolor{commentColor}{HTML}{adb5bd}
\definecolor{mygray}{rgb}{0.5,0.5,0.5}
\definecolor{stringColor}{HTML}{7048e8}
\definecolor{keywordColor}{HTML}{228be6}
\definecolor{backgroundColor}{HTML}{f1f3f5}
\definecolor{borderColor}{HTML}{f1f3f5}
\definecolor{inlineTextColor}{HTML}{495057}
\definecolor{leftRuleColor}{HTML}{868e96}
\definecolor{numbackgroundColor}{HTML}{f1f3f5}
\definecolor{numColor}{HTML}{adb5bd}

\newtcbox{\inlinebox}{enhanced,nobeforeafter,tcbox raise base,boxrule=0pt,top=0.062em,bottom=0.062em,
  right=0.382em,left=0.382em,arc=0.382em,boxsep=0.1em,before upper={\vphantom{dlg}},
  colframe=white,colback=backgroundColor}

\newcommand{\inlinecode}[1] {
  \inlinebox{\lstinline[language=Python, identifierstyle=\color{inlineTextColor}, basicstyle=\color{inlineTextColor}\ttfamily, keywordstyle=\color{inlineTextColor}]{#1}}
}

%\setmonofont{JetBrainsMono NF}[
%    Contextuals = Alternate,
%    Ligatures = TeX,
%]

\lstset{
  backgroundcolor=\color{backgroundColor},   % choose the background color; you must add \usepackage{color} or \usepackage{xcolor}; should come as last argument
  basicstyle=\small\ttfamily,        % the size of the fonts that are used for the code
  breaklines=true,                 % sets automatic line breaking
  commentstyle=\color{commentColor},    % comment style
  extendedchars=true,              % lets you use non-ASCII characters; for 8-bits encodings only, does not work with UTF-8
  firstnumber=1,                % start line enumeration with line 1000
  frame=single,	                   % adds a frame around the code
  frameshape={RYR}{Y}{Y}{RYR},
  keepspaces=true,                 % keeps spaces in text, useful for keeping indentation of code (possibly needs columns=flexible)
  keywordstyle=\color{keywordColor}\textbf,       % keyword style
  language=Python,                 % the language of the code
  morekeywords={*,...},            % if you want to add more keywords to the set
  numbers=left,                    % where to put the line-numbers; possible values are (none, left, right)
  numbersep=1.5em,                   % how far the line-numbers are from the code
  numberstyle=\tiny\color{commentColor}, % the style that is used for the line-numbers
  rulecolor=\color{borderColor},         % if not set, the frame-color may be changed on line-breaks within not-black text (e.g. comments (green here))
  showstringspaces=false,          % underline spaces within strings only
  showtabs=false,                  % show tabs within strings adding particular underscores
  stepnumber=1,                    % the step between two line-numbers. If it's 1, each line will be numbered
  stringstyle=\color{stringColor},     % string literal style
  tabsize=2,
  frame=l,
  framesep=2.12em,
  framexleftmargin=0em,
  fillcolor=\color{numbackgroundColor},
  rulecolor=\color{leftRuleColor},
  numberstyle=\ttfamily\tiny\color{numColor},
}

\newtcolorbox{codebg}[2][]{%
    colback=gray-1,
    %grow to right by=-10mm,
    %grow to left by=-10mm,
    boxrule=0pt,
    boxsep=0pt,
    breakable,
    enhanced jigsaw,
    rounded corners=east,
    arc=8pt,
    borderline west={1pt}{0pt}{gray-3},
    %title={#2\par},
    %colbacktitle={zitat_bg},
    bottomrule=0pt,
}


\newtcbox{\inlinecodee}{on line, boxrule=0pt, boxsep=0pt, top=2pt, left=2pt, bottom=2pt, right=2pt, colback=gray-2, colframe=white, fontupper={\ttfamily \footnotesize}}

\BeforeBeginEnvironment{minted}{\begin{codebg}}%
\AfterEndEnvironment{minted}{\end{codebg}}%

\BeforeBeginEnvironment{inputminted}{\begin{codebg}}%
\AfterEndEnvironment{inputminted}{\end{codebg}}%

\usepackage{minted}
\setminted{
  autogobble=true,
  breakautoindent=true,
  breaklines=true,
  escapeinside=§§,
  fontfamily=tt,
  fontsize=\footnotesize,
  frame=leftline,
  framerule=0pt,
  framesep=0.2em, % sufficient for up to 4 digits
  numbers=left,
  numbersep=0.2em,
  showspaces=false,
  showtabs=false,
  style=vs, % see: https://pygments.org/styles/
  tabsize=2,
  xleftmargin=1.5em,
  % colors
  bgcolor=gray-1,
}
\usemintedstyle{myown}

\tcbuselibrary{minted}

% minted line numbers
\renewcommand{\theFancyVerbLine}{\sffamily
\textcolor{gray-4}{\scriptsize
\oldstylenums{\arabic{FancyVerbLine}}}}


%%%%%%%%%%%%%%%%%%%%%%%%%%%%%%%%%%%%%

\begin{document}

\section*{Halbleiter Notes II: Diode}
gr. diodos -  'Übergang'
\subsection*{Entstehung/ Einstellung des TGG}
\begin{itemize}
\item p und n Halbleiter verbunden
  \item Diffusionsbewegung aufgrund der Konzentrationsunterschiede (mangel an
    Elektronen im p-HL, Überschuss and Elektronen im n-Leiter etc)
  \item Ladungsträger hinterlassen ionisierte, ortsfeste Atomrümpfe der
    Akzeptor-(p-Gebiet) bzw. Donatoratome (n-Gebiet), also ortsfeste Ladungen
    \item rechter Rand des p Gebiets ist negativ geladen
      \item linker Rand des n Gebiets ist positiv geladen
    \item $\rightarrow$ Es entsteht ein elektrisches Feld durch die ionisierten
      und nicht kompensierten Donator- und Akzeptoratome über den
      Übergangsbereich/Grenze 
    \item Dieses Feld (DRIFT) kompensiert/verhindert die weitere Diffusionsbewegung der
      Elektronen aus dem n in das p-Gebiet und der Löcher aus dem p- in das n-Gebiet
    \item Im pn-Übergang entsteht ein Bereich, in dem kein freien Ladungsträger
      (e- oder Löcher) vorhanden sind. In diesem Bereich sind nur die
      ionisierten und ortsfesten Dotieratome vorhanden
    \item Dieser Bereich heißt \emph{Raumladungszone}
\end{itemize}

Über dem pn-Übergang/der Raumladungszone liegt eine Spannung ($\int E \cdot \dif
x$), das \textbf{Diffusionspotential}. Typischerweise im Bereich von $0.6...0.7$ V
\[\phi_i = \underbrace{\frac{k T}{q} \cdot }_{\textrm{Temp.sp.}\, U_T}  \ln{ \frac{N_A \cdot N_D}{n_i^2} }\]

Bei Raumtemperatur: $U_T \approx 26 \, \si{\milli\volt}$

\holine{\textwidth}
\subsection*{Äußere Spannung}
\subsubsection*{Positive Spannung (Durchlass)}
Durch die von Außen angelegte Spannung entsteht ein elektrisches Feld über dem
pn-Übergang, das in
die entgegengesetzte Richtung zum Feld, das durch die ionisierten Dotieratome
entstanden ist, zeigt. Das äußere Feld wirkt also dem inneren entgegen und hebt
damit den Einfluss des inneren auf die Driftbewegung/ die Verhinderung der
Diffusion auf. Die Diffusionsbewegung kann dann wieder stattfinden. Die Löcher
des p-Gebiets können in das n-Gebiet und die Elektronen des n- in das p-Gebiet
diffundieren und dort rekombinieren, 'neue' Ladungsträger kommen dann aus den
neutralen Gebieten. Es fließt ein Strom.

\notebox{

  Bei in Durchlassrichtung angelegter Diodenspannung verringert sich das (innere)
  elektrische Feld über dem pn-Übergang und die Diffusion von Ladungsträgern
  wird nicht mehr durch das elektrische Feld kompensiert.

}
\subsubsection*{Negative Spannung (Sperrung)}
Da das von außen angelegte elektrische Feld in die gleiche Richtung wie das
innere Feld, das die Diffusionsbewegung aufhält, zeigt, verstärkt es dieses und
damit die Driftbewegung und
die Diffusion wird weiter verhindert. Es fließt ein \textbf{Driftstrom aus
  Minoritätsladunsträgern}. Die Elektronen des p-Gebiets und die Löcher des
n-Gebiets driften in das jeweils andere Gebiet. Der resultierende Strom ist
aufgrund der geringen Minoritätsträgerdichten jedoch sehr gering.

\notebox{
In beiden Fällen ändert sich die Breite der Raumladungszone
}

\holine{\textwidth}
\subsection*{Diodengleichung}
\notebox{
  \[ I_D = I_S \cdot (e^{\dfrac{U_{pn}}{U_T}} - 1)  \]
}

Der Diodenstrom ist Temperaturabhängig, der Sperrstrom auch

Die theoretische Kennlinie der Diode weicht für sehr geringe Ströme von der
realen Kennlinie ab, da die Rekombination der Ladungsträger in der
Raumladungszone berücksichtigt werden muss. Führ sehr große Ströme weicht sie
ab, da die Annahme/Vorraussetzung der Schwachen Injektion nicht mehr gegeben ist.
Abhilfe schafft der \emph{Emissionskoeffizient}. Die Funktion wird dann
bereichsweise beschrieben.

\subsection*{Kapazität}
\subsubsection*{Sperrschichtkapazität}
Durch Anlegen einer Spannung ändert sich die Breite der Raumladungszone und
damit die Menge der eingeschlossenen Ladung (unkompensierte Dotieratome).

Ladungsänderung durch Spannungsänderung = Kapazität

\[C_j = \frac{\dif Q_j}{\dif U_{pn}}\]

Diese ist Spannungsabhängig! (Steigt mit zunehmender Spannung $U_{pn}$)
Die Sperrschichtkapazität dominiert im Sperrbereich

\subsubsection*{Diffusionskapazität}
dominiert im Durchlassbereich

\subsection*{Großsignalersatzschaltung}
Spannungsgesteuerte Stromquelle parallel Kapazität (Diffusionskapazität + Sperrschichtkapazität)

\subsubsection*{Transitzeit}
Dioden schalten nicht unendlich schnell von Durchlass in Sperrichtung, da bei
Umpolung zuerst die in der Diode befindliche Ladung (Kapazität, Diff.
oder Sperrschicht) ausgeräumt werden muss. Ein Maß dafür ist die \emph{Transitzeit}.

\subsection*{Kleinsignalersatzschaltung}

Beschreibt das Verhalten bei \textbf{kleinen Aussteuerungen um einen festen Arbeitspunkt}
Dazu annäherung der Werte um den Arbeitspunkt durch eine Gerade, dessen Steigung
der Leitwert $$g_D = \frac{\dif I_D}{\dif U_{pn}}$$ an der Stelle des
Arbeitspunktes ist. Damit ist $$i_D = g_D \cdot u_{pn}$$

\subsubsection*{Me-HL-Kontakte}
Metall-Halbleiterkontakte können Shottky-Kontakte sein, wenn die Austrittsarbeit
der Elektronen im Halbleiter geringer ist als die im Metall oder Ohmsche
Kontakte sein, wenn die Austrittsarbeit der Elektronen im Halbleiter höher ist
als die im Metall. Vom Material mit der geringeren Austrittsarbeit gelangen
Elektronen einfacher in das Material mit höherer Austrittsarbeit.
Ohmsche Kontakte können an Metall-Halbleiter-Kontaktstellen erreicht werden,
indem man den Halbleiter an den Kontaktstellen sehr hoch dotiert, da so die
Raumladungszone der Metall-HL-Diode sehr klein wird und Elektronen diese einfach
durchtunneln können. Dadurch entsteht ein ohmscher Charakter des Übergangs.

\end{document}
