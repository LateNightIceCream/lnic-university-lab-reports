\documentclass[a4paper, 12pt]{article}

%%% SST LAB PROTOCOLL PREAMBLE
%%% 2019
%%%%%%%%%%%%%%%%%%%%%%%%%%%%%%%


%%% PACKAGES
%%%%%%%%%%%%%%%%%%%%%%%%%%%

\usepackage[ngerman]{babel}

\usepackage[utf8]{inputenc}
\usepackage{amsmath}
\usepackage{pgfplots}
\usepackage{tikz}
\usepackage[many]{tcolorbox}
\usepackage{graphicx}
\graphicspath{ {./graphics/} }
\usepackage{pdfpages}
\usepackage{dashrule}
\usepackage{float}
\usepackage{siunitx}
\usepackage{trfsigns}
\usepackage{booktabs}
\usepackage[european]{circuitikz}
\usepackage{tcolorbox}

%%% DOCUMENT GEOMETRY
%%%%%%%%%%%%%%%%%%%%%%%%%%%

\usepackage{geometry}
\geometry{
 a4paper,
 total={0.6180339887498948\paperwidth,0.6180339887498948\paperheight},
 top = 0.1458980337503154\paperheight,
 bottom = 0.1458980337503154\paperheight
 }
\setlength{\jot}{0.013155617496424828\paperheight}
\linespread{1.1458980337503154}

\setlength{\parskip}{0.013155617496424828\paperheight} % paragraph spacing


%%% COLORS
%%%%%%%%%%%%%%%%%%%%%%%%%%%

\definecolor{red1}{HTML}{f38181}
\definecolor{yellow1}{HTML}{fce38a}
\definecolor{green1}{HTML}{95e1d3}
\definecolor{blue1}{HTML}{66bfbf}
\definecolor{hsblue}{HTML}{00b1db}
\definecolor{hsgrey}{HTML}{afafaf}

%%% CONSTANTS
%%%%%%%%%%%%%%%%%%%%%%%%%%%
\newlength{\smallvert}
\setlength{\smallvert}{0.0131556\paperheight}


%%% COMMANDS
%%%%%%%%%%%%%%%%%%%%%%%%%%%

% differential d
\newcommand*\dif{\mathop{}\!\mathrm{d}}

% horizontal line
\newcommand{\holine}[1]{
  	\begin{center}
	  	\noindent{\color{hsgrey}\hdashrule[0ex]{#1}{1pt}{3mm}}\\%[0.0131556\paperheight]
  	\end{center}
}

% mini section
\newcommand{\minisec}[1]{ \noindent\underline{\textit {#1} } \\}

% quick function plot
\newcommand{\plotfun}[3]{
  \vspace{0.021286\paperheight}
  \begin{center}
    \begin{tikzpicture}
      \begin{axis}[
        axis x line=center,
        axis y line=center,
        ]
        \addplot[draw=red1][domain=#2:#3]{#1};
      \end{axis}
    \end{tikzpicture}
  \end{center}
}

% box for notes
\newcommand{\notebox}[1]{

\tcbset{colback=white,colframe=green1!100!black,title=Note!,width=0.618\paperwidth,arc=0pt}

 \begin{center}
  \begin{tcolorbox}[]
   #1 
  \end{tcolorbox}
 
 \end{center} 
 
}

% box for equation
\newcommand{\eqbox}[2]{
	
	\tcbset{colback=white,colframe=green1!100!black,title=,width=#2,arc=0pt}
	
	\begin{center}
		\begin{tcolorbox}[ams align*]
				#1
		\end{tcolorbox}
		
	\end{center} 
	
}
% END OF PREAMBLE

%%%%%%%%%%%%%%%%%%%%%%%%%%%%%%%%%%%%%

\begin{document}

\section*{Halbleiter Notes}

\subsection*{Dotierung}
\textbf{ n-Dotierung},z.B. Phosphor (5-wertig) in Silizium:
\begin{itemize} 
\item Donatoratome:
\item Ein Elektron mehr als in den Bindungen zu den Siliziumatomen notwendig:
\item 5tes Elektron der Phosphoratome hat Ionisierungsenergie leicht unterhalb des Leitungsbandes des Gitters:
\item Elektron geht bereits bei Raumtemperatur in das LB über und trägt zur (Elektronen-)Leitfähigkeit bei:
\item Eigenleitungsdichte wird erst bei höheren Temperaturen ausgelöst, die Dichte der durch die Ionisierung entstandenen Elektronen überwiegt daher:
\item Dichte der Dotieratome ist daher etwa der Dichte der Freien Elektronen.z.B. Phosphor (5-wertig) in Silizium:
 \item Thermodyn. Gleichgewicht .. Generierungsrate = Rekombinationsrate..Massenwirkungsgesetz
\end{itemize}

\noindent \textbf{ p-Dotierung}, z.B. Bor (3-wertig) in Silizium:
\begin{itemize}
\item Akzeptoratome:
\item Ein Elektron weniger als in den Bindungen zu den Siliziumatomen notwendig:
\item Ionisierungsenergie der Boratome leicht überhalb des Valenzbandes des
  Halbleitergitters
\item Elektron des Valenzbandes kann bereits bei Raumtemperatur (Energiezufuhr)
  zum Elektron des Boratoms werden und es damit negativ ionisieren
\item Es bleibt ein Loch im Valenzband übrig
\item Loch trägt im Valenzband wie ein Elektron im Leitungsband zur
  Leitfähigkeit des Halbleiters bei
\item Rest siehe n-Dotierung
\end{itemize}

\subsection*{Massenwirkungsgesetz}

Im thermodynamischen Gleichgewicht (Rekombinationsrate = Generierungsrate) gilt:
$$ n_i^2 = p_0 \cdot n_0  $$

Da die Zunahme der Anzahl von z.B. Elektronen im Halbleiter (über die
Gleichgewichtslage hinaus) eine Erhöhung der Rekombinationsrate nach
\[ R(T) = r(T) \cdot n \cdot p\]
hervorruft, die Löcherdichte jedoch nicht geändert wurde, wird auch die Anzahl
der Löcher/ Die Löcherdichte geringer (Rekombinationsrate steigt durch Erhöhung der einen
Ladungsträgerart $\rightarrow$ andere Ladungsträgerart wird stärker 'verbraucht' )

\notebox{Massenwirkungsgesetz: Eine Zunahme der einen Ladungsträgerart/dichte
  führt zur Abnahme der anderen (im thermodyn. Gleichgewicht)}

\subsection*{Ferminiveau}

Lage des Ferminiveaus (relativ zum Intrinsien-Niveau/ Mitte zwischen beiden Bandkanten)

\noindent \textbf{n-Leiter}:
\[ W_F = kT\cdot \ln{\frac{N_D}{n_i}}\]
\textbf{p-Leiter}:
\[ W_F = -kT\cdot \ln{\frac{N_A}{n_i}}\]

\notebox{Das Ferminiveau liegt umso näher an der Leitungsbandkante, je stärker
  der HL mit Donatoratomen dotiert ist (analog für p-Dotierung)}

Der Begriff des Ferminiveaus ist nur im thermodynamischen Gleichgewicht
sinnvoll, jedoch kann man auch bei Störungen des Gleichgewichts das Ferminiveau
als Bezugspunkt verwenden. Das anlegen einer Spannung führt demnach zu einer
Verschiebung des Ferminiveaus. Liegt dieses nicht waagerecht, so fließt ein Strom
durch den Halbleiter.

\subsection*{Elektronen- und Löcherstrom}

Die Gesamtstromdichte im HL setzt sich aus Elektronen- und Löcherstromdichte
zusammen
\[ j = j_n + j_p \]

Elektronen- und Löcherstrom bestehen im HL aus dem \emph{Drift-} und dem
\emph{Diffusionsstrom}.\\

\textbf{Driftstrom}:
\[ j_{Drift, n} = -q \cdot n \cdot v_n\]
\[ j_{Drift, p} = q \cdot n \cdot v_p\]
\[ j_{Drift, gesamt} = \sigma \cdot E\]
\[\sigma = q [\mu_n\cdot n + \mu_p \cdot p]\]

\textbf{Diffusionsstrom}:\\
Durch \textbf{thermische} Konzentrationsausgleichsvorgänge von Teilchen. Diese
Teilchen sind im HL Löcher oder Elektronen, somit also bewegte Ladung und damit
ein el. Strom. Dieser Diffusionsstrom ist proportional zum Gradienten der Ladungsträgerkonzentration
\[ j_{Diff, n} = q \cdot D_n \cdot \frac{\dif n}{\dif x}\]

Wobei $D_n$ der Diffusionskoeffizient (Eisteinbeziehung) ist. (analog für Löcher
mit umgekehrten Vorzeichen)\\

Die gesamten Stromdichten sind damit:
\eqbox{
 j_n = q \cdot E \cdot \mu_n \cdot n + q \cdot D_n \cdot \frac{\dif n}{\dif x}\\
 j_p = q \cdot E \cdot \mu_p \cdot p - q \cdot D_p \cdot \frac{\dif p}{\dif x}
}{0.618\textwidth}


\notebox{
\noindent {\bf Driftstrom:} Durch el. Feld verursacht, steigt mit der Größe des
el. Feldes und der Ladungsträgerdichte\\ \\
{\bf Diffusionsstrom:} Durch Konzentrationsunterschiede der Ladungsträgerdichte
verursacht, vom el. Feld unabhängig.\\
}

\holine{\textwidth}

\subsection*{Beweglichkeit}
Die Beweglichkeit $\mu$ beschreibt das Streuverhalten bzw. die \emph{Ungehindertheit
driftender Ladungsträger}.
\notebox{
\[ v_D = \mu \cdot E \]
}
Für Elektronen und Löcher gibt es im HL unterschiedliche Beweglichkeiten
($\mu_n, \mu_p$). Die Beweglichkeit ist temperatur- und feldstärkeabhängig.
Die Driftgeschwindigkeit geht bei hohen Feldstärken in Sättigung, weshalb die
Beweglichkeit nicht konstant ist.

\subsection*{Injektionen}
Betrachtet wird der Fall der Störung des thermodynamischen Gleichgewichts des Halbleiters

\subsubsection*{Schwache Injektion}
(Mit Beispiel eines n-Leiters)
Die Minoritätsträgerdichte ($p_n$) ist gegenüber der Majoritätsträgerdichte
($n_n$) sehr gering. Die Majoritätsträgerdichte ist nur leicht höher als die
Majoritätsträgerdichte im Gleichgewichtszustand ($n_0$)
$$n_n \approx n_0,\,\, n_n > > p_n$$

\subsubsection*{Starke Injektion}
Majoritätsträgerdichte und Minoritätsträgerdichte liegen deutlich über den Gleichgewichtswerten.
$$n_n \approx p_n, \,\, n_n > > N_D$$

\end{document}
