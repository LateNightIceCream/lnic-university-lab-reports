\documentclass[a4paper, 12pt]{article}

%%% SST LAB PROTOCOLL PREAMBLE
%%% 2019
%%%%%%%%%%%%%%%%%%%%%%%%%%%%%%%


%%% PACKAGES
%%%%%%%%%%%%%%%%%%%%%%%%%%%

\usepackage[ngerman]{babel}

\usepackage[utf8]{inputenc}
\usepackage{amsmath}
\usepackage{pgfplots}
\usepackage{tikz}
\usepackage[many]{tcolorbox}
\usepackage{graphicx}
\graphicspath{ {./graphics/} }
\usepackage{pdfpages}
\usepackage{dashrule}
\usepackage{float}
\usepackage{siunitx}
\usepackage{trfsigns}
\usepackage{booktabs}
\usepackage[european]{circuitikz}
\usepackage{tcolorbox}

%%% DOCUMENT GEOMETRY
%%%%%%%%%%%%%%%%%%%%%%%%%%%

\usepackage{geometry}
\geometry{
 a4paper,
 total={0.6180339887498948\paperwidth,0.6180339887498948\paperheight},
 top = 0.1458980337503154\paperheight,
 bottom = 0.1458980337503154\paperheight
 }
\setlength{\jot}{0.013155617496424828\paperheight}
\linespread{1.1458980337503154}

\setlength{\parskip}{0.013155617496424828\paperheight} % paragraph spacing


%%% COLORS
%%%%%%%%%%%%%%%%%%%%%%%%%%%

\definecolor{red1}{HTML}{f38181}
\definecolor{yellow1}{HTML}{fce38a}
\definecolor{green1}{HTML}{95e1d3}
\definecolor{blue1}{HTML}{66bfbf}
\definecolor{hsblue}{HTML}{00b1db}
\definecolor{hsgrey}{HTML}{afafaf}

%%% CONSTANTS
%%%%%%%%%%%%%%%%%%%%%%%%%%%
\newlength{\smallvert}
\setlength{\smallvert}{0.0131556\paperheight}


%%% COMMANDS
%%%%%%%%%%%%%%%%%%%%%%%%%%%

% differential d
\newcommand*\dif{\mathop{}\!\mathrm{d}}

% horizontal line
\newcommand{\holine}[1]{
  	\begin{center}
	  	\noindent{\color{hsgrey}\hdashrule[0ex]{#1}{1pt}{3mm}}\\%[0.0131556\paperheight]
  	\end{center}
}

% mini section
\newcommand{\minisec}[1]{ \noindent\underline{\textit {#1} } \\}

% quick function plot
\newcommand{\plotfun}[3]{
  \vspace{0.021286\paperheight}
  \begin{center}
    \begin{tikzpicture}
      \begin{axis}[
        axis x line=center,
        axis y line=center,
        ]
        \addplot[draw=red1][domain=#2:#3]{#1};
      \end{axis}
    \end{tikzpicture}
  \end{center}
}

% box for notes
\newcommand{\notebox}[1]{

\tcbset{colback=white,colframe=green1!100!black,title=Note!,width=0.618\paperwidth,arc=0pt}

 \begin{center}
  \begin{tcolorbox}[]
   #1 
  \end{tcolorbox}
 
 \end{center} 
 
}

% box for equation
\newcommand{\eqbox}[2]{
	
	\tcbset{colback=white,colframe=green1!100!black,title=,width=#2,arc=0pt}
	
	\begin{center}
		\begin{tcolorbox}[ams align*]
				#1
		\end{tcolorbox}
		
	\end{center} 
	
}
% END OF PREAMBLE

%%%%%%%%%%%%%%%%%%%%%%%%%%%%%%%%%%%%%

\begin{document}

\section*{Halbleiter Notes III: Bipolartransistor}

en. Transistor ... \textbf{Trans}fer-re\textbf{sistor}\\

$\rightarrow$ Durch Anlegen einer Spannung steuerbarer Widerstand\\

\noindent \textbf{Bipolar}: Funktion durch \textbf{beide Ladungsträgerarten}\\

\noindent 3 Halbleitergebiete, dotierfolge entweder \textbf{NPN} oder \textbf{PNP}

\begin{itemize}
  \item Der Pfeil im Schaltsymbol gibt die \emph{technische Stromrichtung im
      Normalbetrieb} an
\end{itemize}

\subsection*{Funktion}

Am Beispiel der \emph{npn-Dotierfolge}
\begin{itemize}
  \item Ist die von außen anliegende Spannung $U_{BE} = 0$, ist sind alle
    pn-Übergänge gesperrt und es fließt kein Strom durch den Transistor.
  \item Spannung von etwa $U_{BE} = 0.7 \si{\volt}$ an den Basis-Emitter
    Übergang
    \item Basis-Emitter-Übergang ist in Durchlassbetrieb
  \item Diffusion von Löchern aus dem p- (basis) in das n- (Emitter) Gebiet
    (vgl. Diode: Verringerung der Breite der Raumladungszone durch äußeres Feld,
    Diffusionsspannung wird kompensiert, Diffusion überwiegt ggü
    entgegengesetztem Drift, etc.)
  \item \emph{Rekombination} der Löcher aus der Basis mit den Elektronen im Emitter 
  \item  \textbf{Elektronen des Emitters} diffundieren ebenfalls in die Basis,
    \emph{rekombinieren jedoch nicht}, da die Basis sehr kurz ist
  \item Durch äußere Spannung ist das Elektrische Feld der Raumladungszone vom
    Basis-Kollektorübergang so gerichtet, dass die Elektronen des Emitters sich
    in Richtung Kollektor bewegen (\textbf{Drift})
  \item Es fließt ein \emph{Elektronenstrom} zwischen Kollektor und Emitter
  \item Normalbetrieb: Basis-Emitter-Übergang in Durchlass,
    Basis-Kollektor-Übergang in Sperrichtung
  \end{itemize}

\subsection*{Gleichungen}
Annahmen: Vom Kollektor in die Basis injizierten Löcher werden vernachlässigt;
Die Elektronen die nicht von der Basis in den Kollektor abgesaugt werden sondern
in der Basis rekombinieren werden vernachlässigt

\subsubsection*{Kollektorstrom:}
\eqbox{
  I_C = I_S \cdot (e^{\dfrac{U_{BE}}{U_T}}-1)
}{0.618\textwidth}

Transfersättigungsstrom:
\eqbox{
  I_S = \frac{A \cdot q \cdot D_{nB} \cdot n_{B0}}{x_B}
}{0.618\textwidth}

($I_S$ typ. $10^{-17} \, \si{\ampere}$)

\subsubsection*{Basisstrom:}
Die Eingangskennlinie ($I_B=f(U_{BE})$) ergibt sich als einfache Diodenkennlinie

\eqbox{
  I_B = \frac{A \cdot q \cdot D_{pE} \cdot p_{E0}}{L_{pE}} \cdot (e^{\dfrac{U_{BE}}{U_T}}-1)
}{0.618\textwidth}

Durchlass ungefähr bei $U_{BE} \approx 0.7 \, \si{\volt}$

\minisec{Mit Stromverstärkung unten:}
\eqbox{
  I_B = \frac{I_S}{B_N}\cdot (e^{\dfrac{U_{BE}}{U_T}}-1)
}{0.618\textwidth}

\subsubsection*{Emitterstrom}
$\rightarrow$ Kirchhoff

\subsubsection*{Stromverstärkung}
Verhältnis von \emph{Kollektor- zu Basisstrom}
\[ B_N = \frac{I_C}{I_B} \] (Index N für Normalbetrieb)

\eqbox{
  B_N = \frac{D_{nB} \cdot N_{DE} \cdot L_{pE}}{D_{pE} \cdot N_{AB} \cdot x_B}
}{0.618 \textwidth}

$N_{DE}, N_{AB}$...Dotierungsdichten\\
$L_{pE}$...Diffusionslänge\\


Der Stromverstärkungsfaktor ist nicht konstant sondern für sehr kleine und sehr
große Kollektorströme vom \textbf{Kollektorstrom abhängig}

\subsection*{Inversbetrieb}
Vertauschen von Kollektor und Emitter.\\
Basis-Kollektor-Übergang ist in Durchlassrichtung, Basis-Emitter-Übergang ist in Sperrrichtung.

Die unterschiedliche Verhaltensweise im inversbetrieb ist auf die
unterschiedlichen Dotierungen/Weiten der einzelnen Bereiche zurückzuf+hren.
Dadurch ändert sich die Stromverstärkung.

$$B_I = \frac{I_E}{I_B} = \textrm{siehe oben}$$

Die Stromverstärkung im Inversbetrieb ist geringer als im Normalbetrieb

\subsection*{Sättigungsbetrieb}
Beide pn-Übergänge in Durchlassrichtung\\
\textbf{Langsameres Schaltverhalten}

Näherung: $U_{CE,sat} \approx 0.1 \, \si{\volt}$

\subsection*{Ausgangskennlinienfeld}
Darstellung des Kollektorstroms in Abhängigkeit von der
Kollektor-Emitter-Spannung für verschiedene Werte des Basisstroms.\\

Das Kennlininenfeld beginnt bei fortlaufender Kollektorspannung mit dem
Transistor im Sättigungsbetrieb, da die Basis-Emitter-Spannung $> 0V$ und die
Basis-Kollektorspannung ebenfalls $> 0V$ ist; Beide pn-Übergänge sind in
Durchlassrichtung gepolt, dadurch Abnahme des Kollektorstroms mit kleiner
werdender CE-Spannung; Dies ist nicht der Normalbetriebsfall und die
Stromverstärkung $B_N$ gilt daher nicht. Da gilt
\[U_{BC} = U_{BE} - U_{CE}\],
sinkt die Basis-Kollektorspannung bei gleichbleibender Basis-Emitter-Spannung
mit sinkender Kollektor-Emitter-Spannung, wodurch der Transistor in Sättigung
geht, wenn $U_{BC}$ positiv wird (i guess)\\

Der andere Bereich heißt \emph{aktiver Vorwärtsbetrieb}


\subsection*{Basisweitenmodulation (Early-Effekt)}
Im theoretischen Ausgangskennlinienfeld ist der Kollektorstrom im aktiven
Vorwärtsbetrieb unabhängig von der Kollektor-Emitter-Spannung (waagerechte). In
der Praxis steigt der Strom jedoch mit steigender Kollektorspannung\\

\minisec{Ursache:}
Die Änderung der Kollektor-Emitter-Spannung und damit die \textbf{Änderung der
Basis-Kollektor-Spannung} (Sperrspannung) führt zu einer \textbf{Änderung der Weite der
Basis-Kollektor-Raumladungszone}. Die effektive Basisweite ändert sich und damit
ändert sich der Kollektorstrom. \\

Es ergibt sich eine Gerade im aktiven
Vorwärtsbetrieb, dessen Nullstelle die \emph{Early-Spannung $U_{AN}$} ist. Die
Steigung der Geraden ist
\[\frac{I_C|_{U_{BC} = 0}}{U_{AN}}\]

\subsection*{Modellierung}
\subsubsection*{Großsignalersatzschaltbild}
Eingangskreis: Diode\\
Ausgangskreis: Gesteuerte Stromquelle\\

Transportmodell: Zusammenfassung von Normal- und Inversbetrieb $\rightarrow$
zwei Dioden/pn-Übergänge (BC, BE) und zwei Basisströme\\

Für das dynamische Verhalten können noch die Kapazitäten hinzugefügt werden\\

\subsubsection*{Schaltverhalten}
Da der Basis-Emitter-Übergang eine Diode darstellt, muss zuerst die in der Basis
befindliche Ladung ausgeräumt werden, bevor die Diode in Sperrrichtung gerät.\\

Trennt man die Basis-Spannungsquelle ab, kann die in der Basis befindliche
Ladung nur durch Rekombination entfernt werden, wodurch der Ausschaltvorgang
deutlich verlängert wird.\\

Schaltzeitverlängerung im Sättigungsbetrieb.\\

\minisec{Schottky-Transistor}
Durch parallelschalten einer Schottkydiode (Me-HL-Übergang) zum
Basis-Kollektorübergang können der Sättigungsbetrieb 
und die Schaltzeit beim Ausschalten verringert werden  

\subsubsection*{Kleinsignalersatzschaltbild}
Linearisierung der Großsignalersatzschaltung um den Arbeitspunkt (Wie bei der
Diode!)\\

\notebox{
  Das Kleinsignalersatzschaltbild gilt nur für den angegebenen, festen
  Arbeitspunkt herum bei Aussteuerung mit kleinen Signalen. 
}

Abhängigkeit des Kollektorstroms von der Eingangsspannung: Steigung der gerade:
Leitwert 'Steilheit' $g_m = \frac{q}{k \cdot T} \cdot I_{C,A}$\\

Abhängigkeit des Kollektorstroms von der Kollektor-Emitter-Spannung:
Ausgangsleitwert: $g_0 = \frac{I_{C,A}}{U_{AN} + U_{CE,A}}$\\

Änderung des Kollektorstroms durch Änderung des Basisstroms:
Kleinsignalstromverstärkung $\beta_N = \frac{i_C}{i_B} = \frac{\dif I_C}{\dif
  I_B}$\\

Änderung des Basisstroms bei Änderung der BE-Spannung: Eingangsleitwert $g_\pi =
\frac{I_{C,A}}{\beta_N \cdot U_T}$\\

Änderung des Basisstroms bei Änderung der CE-Spannung: Rückwärtssteilheit: ist
vernachlässigbar! = 0, für Rest siehe buch

\subsubsection*{Frequenzverhalten}
Stromverstärkung ist frequenzabhängig! Die \textbf{Transitfrequenz} gibt die
Frequenz an, bei der die Stromverstärkung nur noch 1 ist.
Die parasitären Kapazitäten führen zu einer Abnhame der Transitfrequenz

\end{document}
