\documentclass[a4paper, 12pt]{article}

\setlength\parindent{0pt}

\usepackage{amsmath}
\usepackage{pgfplots}
\usepackage{tikz}
\usepackage{tcolorbox}

\definecolor{red1}{HTML}{f38181}
\definecolor{yellow1}{HTML}{fce38a}
\definecolor{green1}{HTML}{95e1d3}
\definecolor{blue1}{HTML}{66bfbf}

\newcommand{\holine}{
  \noindent\rule{\textwidth}{0.618pt}\\[0.021286\paperheight]
}

\newcommand{\minisec}[1]{ \underline{\textit {#1} } \\[0.021286\paperheight]}

\newcommand{\plotfun}[3]{
  \vspace{0.021286\paperheight}
  \begin{center}
    \begin{tikzpicture}
      \begin{axis}[
        axis x line=center,
        axis y line=center,
        ]
        \addplot[draw=red1][domain=#2:#3]{#1};
      \end{axis}
    \end{tikzpicture}
  \end{center}
}

\tcbset{colback=white,colframe=red1!100!black,title=Note!,width=0.618\paperwidth,arc=0pt}
\newcommand{\notebox}[1]{

 \begin{center}
  \begin{tcolorbox}[]
   #1 
  \end{tcolorbox}
 
 \end{center} 
 
}

\begin{document}

\section*{Oscilloscopes, an overwiew}

Oscilloscopes are useful for displaying specific numerical measurements as
well as depicting graphical representations of AC Waveforms.\\
  
\holine 

\subsection*{ General }

Oscilloscope screens are divided into vertical divisions and horizontal
divisions where the vertical axis is showing the current value of the
measured/calculated quantity (voltage, power, etc.) in the current domain (time,
frequency), shown by the horizontal axis. The axes divisions can be scaled to fit the appropriate value of
measurement. The scaling is given in unit/division so the
interpretation of an oscilloscope graph/plot requires knowledge of the divison scaling. 

\notebox{

  In order to successfully interpret and analyze an oscilloscope picture, one has to know:\\
  \begin{itemize}
    \item[+]{Horizontal scaling including Units}
    \item[+]{Vertical scaling including Units}
    \item[+]{Domain (time or frequency)}\\

    \small{Do the calculations first!}
  \end{itemize}
}

\subsection*{ Layout }

Scopes can have multiple channels to display multiple waveforms at the same
time. There is a vertical group for each channel to scale the signals
individually. There is a horizontal group to scale the horizontal axis. There is
a Trigger Section. There is a function group, within which there is a selection option that can be pushed and rotated to
make menu selections on the scope.\\

There might be Test Points (PROBE COMP), square metal contacts with a hole,
which continuously generate a test signal (i.e. a square wave).

\subsection*{Probe Attenuation}
Scope probes have a switch to change the \emph{attenuation}.\\

\notebox{ Attenuation is the act of \emph{reducing something}}.

Attenuation values:\\

\textbf{1X} : Signal is not attenuated, the probe will deliver 1V to the scope for every
1V experienced at the grabber.\\

\textbf{10X}: Signal is attenuated 10 times, the probe will delive 1/10th of 1V to the
scope for every 1V experienced at the grabber.\\

\notebox{ 10X attenuation means dividing by 10, not multiplying }

Attenuation is needed when the measured voltage exeeds the limits of the
oscilloscope. Attenuation on the probe and on the scope both have to match for proper results.

\subsection*{Couple}
Coupling is a means of displaying the waveform with or without removing its DC
Offset (and grounding it).\\

There are:

\begin{itemize}
  \item[-]{ DC Coupling }
    $\rightarrow $DC Offset included (AC and DC)
  \item[-]{ AC Coupling }
    $\rightarrow $DC Offset removed (\textbf{zero centered} -- Pure AC)
  \item[-]{ Grounding (GND)}
    $\rightarrow $flatlines the display
\end{itemize}

\subsection*{Positioning}
Positioning knob in the vertical group:\\

changing the vertical axis position of the waveform
indicated by arrow with the channel number on the left.

\subsection*{Trigger}
The trigger value will tell the oscilloscope when to measure / to trigger the
measurement of the channel, i.e. when set to 1.2V, the oscilloscope will wait
until the triggering event (1.2V value) is reached and will then sample the
waveform for the set amount of time. The trigger is indicated by an arrow on the
right of the screen.
A regularily occuring trigger event will result in regular occuring data.

\notebox{

  The trigger basically tells the scope when to start on the left side.
  
}

Waveforms are not always periodic though, so there is a button on the
oscilloscope:\\
\textbf{ RUN/STOP }\\
that captures a momentary picture of the waveform \\

Triggering also works horizontally, shifting the measured waveform left or right.

\holine

You should set the oscilloscope up with proper settings before measuring the
waveform. Try to get as much information about the waveform(s) under inspection
as possible and \emph{then} set up the oscilloscope.

\notebox{\textbf{Do the calculations first!}}

\pagebreak

\end{document}