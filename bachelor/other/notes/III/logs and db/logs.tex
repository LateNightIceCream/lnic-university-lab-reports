\documentclass[a4paper, 12pt]{article}

\setlength\parindent{0pt}

\usepackage{amsmath}
\usepackage{pgfplots}
\usepackage{svg}
\usepackage{graphicx,import}

\newcommand{\holine}{
  \noindent\rule{\textwidth}{0.618pt}
}
\newcommand{\inklude}[1]{%
    \def\svgwidth{\columnwidth}
    \import{./figures/}{#1.pdf_tex}
}
\begin{document}

\section*{Logarithms}

if
$$ b^x = N $$
then
$$ x = \log_b{N} $$

\begin{center}
Logarithms return exponents.
\end{center}

$$ \log_{10}{100} = 2 $$
because $10^2 = 100$

\holine

\begin{align*}
  \log{0} = \textrm{undefined} (\rightarrow -\infty)\\
  \log{a} = \textrm{undefined} ; a<1\\
  \log{1} = 0 \\
  \log{(a \cdot b} = \log(a) + \log(b)\\
  \log{\frac{a}{b}} = \log(a) - \log(b)\\
  \log{a^b} = b\cdot\log(a)\\
\end{align*}

\holine

\begin{center}
\begin{tikzpicture}
  \begin{axis}
    \addplot[color=red]{log10(x)};
  \end{axis}
\end{tikzpicture}
\end{center}

Logarithms allow for a graphical representation of both very small and very
large values in a plot. Using a linearily scaled x- (or y-)axis results in fixed
$\Delta x$ ($\Delta y$) which also results in a fixed physical distance (i.e. on
a piece of paper or a screen) between each axis ticks. So as the range of values
increases, the space needed for the plot increases linearily.\\

With logarithmic scales, each tick is one power of 10 greater than its
predecessor, i.e. $10^1 = 10, 10^2 = 100, 10^3 = 1000$.\\

When choosing to plot raw data, say i.e. the datum 500, with a log-scale of i.e. base 10,
one has to take the log of each datum to find the distance of the point from 0.

\[ \log{500} \approx 2.7 \]

which means that the datum 500 is 2.7 units from the 0 point. This is also where
the name 'logarithmic' scale comes from.
You get the exponent of 10 on the axis you want to plot the value on.

\holine

\subsection*{Gain and dB}
Systems are divided into active and passive.\\

\textit{Active} systems produce an output that is \textit{greater} than their
input. They require external supply for amplification\\

\textit{Passive} systems produce an output that is \textit{less} than their
input\\

\inklude{schnibibean}

The Gain of active and passive Systems make use of logarithmic units.\\

The standard (logarithmic) unit for gain is the Bel.\\

$$\textrm{GAIN/Bel} = \log{\frac{P_{out}}{P_{in}}}$$

$$10 \textrm{dB} = 1 Bel$$ so

$$\textrm{GAIN/deciBel} = 10 \cdot \log{\frac{P_{out}}{P_{in}}}$$\\

It sometimes makes more sense to use voltage or current in order to calculate
gain, so express power accordingly:

$$\textrm{GAIN/dB} = 10 \cdot \log{\frac{\dfrac{V_1^2}{R}}{\dfrac{V_2^2}{R}}} =
20 \cdot \log{\frac{V_1}{V_2}}$$ that is if we assume Resistance to be the same!

\underline{Examples:}\\
$$V_{in} = 5V, V_{out} = 12V, V_{out}/V_{in} = 2.4$$
$$\rightarrow 20 \log{\frac{12V}{5V}} = 7.6 dB$$\\

$$V_{in} = 0.005V, V_{out} = 12V, V_{out}/V_{in} = 2400$$
$$\rightarrow 20 \log{\frac{12V}{0.005V}} = 67.6 dB$$

Relation to Active/Passive Systems:\\[0.5cm]

Passive Systems: Output $<$ Input $\rightarrow$ - GAIN\\
Active Systems: Output $>$ Input $\rightarrow$ + GAIN\\

For filters, the critical frequency describes the situation where output Power
is half of input Power
$$ V_{out} = V_{in} / \sqrt{2} \rightarrow 1/2 \cdot P_{max}, GAIN: -3 dB$$
$$ V_{out} = V_{in} / 10 \rightarrow 1/100 \cdot P_{max}, GAIN: -20 dB$$

\end{document}