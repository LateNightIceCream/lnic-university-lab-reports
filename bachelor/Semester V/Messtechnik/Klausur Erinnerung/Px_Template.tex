\documentclass[a4paper, 12pt]{article}

%%% SST LAB PROTOCOLL PREAMBLE
%%% 2019
%%%%%%%%%%%%%%%%%%%%%%%%%%%%%%%


%%% PACKAGES
%%%%%%%%%%%%%%%%%%%%%%%%%%%

\usepackage[ngerman]{babel}

\usepackage[utf8]{inputenc}
\usepackage{amsmath}
\usepackage{pgfplots}
\usepackage{tikz}
\usepackage[many]{tcolorbox}
\usepackage{graphicx}
\graphicspath{ {./graphics/} }
\usepackage{pdfpages}
\usepackage{dashrule}
\usepackage{float}
\usepackage{siunitx}
\usepackage{trfsigns}
\usepackage{booktabs}
\usepackage[european]{circuitikz}
\usepackage{tcolorbox}

%%% DOCUMENT GEOMETRY
%%%%%%%%%%%%%%%%%%%%%%%%%%%

\usepackage{geometry}
\geometry{
 a4paper,
 total={0.6180339887498948\paperwidth,0.6180339887498948\paperheight},
 top = 0.1458980337503154\paperheight,
 bottom = 0.1458980337503154\paperheight
 }
\setlength{\jot}{0.013155617496424828\paperheight}
\linespread{1.1458980337503154}

\setlength{\parskip}{0.013155617496424828\paperheight} % paragraph spacing


%%% COLORS
%%%%%%%%%%%%%%%%%%%%%%%%%%%

\definecolor{red1}{HTML}{f38181}
\definecolor{yellow1}{HTML}{fce38a}
\definecolor{green1}{HTML}{95e1d3}
\definecolor{blue1}{HTML}{66bfbf}
\definecolor{hsblue}{HTML}{00b1db}
\definecolor{hsgrey}{HTML}{afafaf}

%%% CONSTANTS
%%%%%%%%%%%%%%%%%%%%%%%%%%%
\newlength{\smallvert}
\setlength{\smallvert}{0.0131556\paperheight}


%%% COMMANDS
%%%%%%%%%%%%%%%%%%%%%%%%%%%

% differential d
\newcommand*\dif{\mathop{}\!\mathrm{d}}

% horizontal line
\newcommand{\holine}[1]{
  	\begin{center}
	  	\noindent{\color{hsgrey}\hdashrule[0ex]{#1}{1pt}{3mm}}\\%[0.0131556\paperheight]
  	\end{center}
}

% mini section
\newcommand{\minisec}[1]{ \noindent\underline{\textit {#1} } \\}

% quick function plot
\newcommand{\plotfun}[3]{
  \vspace{0.021286\paperheight}
  \begin{center}
    \begin{tikzpicture}
      \begin{axis}[
        axis x line=center,
        axis y line=center,
        ]
        \addplot[draw=red1][domain=#2:#3]{#1};
      \end{axis}
    \end{tikzpicture}
  \end{center}
}

% box for notes
\newcommand{\notebox}[1]{

\tcbset{colback=white,colframe=green1!100!black,title=Note!,width=0.618\paperwidth,arc=0pt}

 \begin{center}
  \begin{tcolorbox}[]
   #1 
  \end{tcolorbox}
 
 \end{center} 
 
}

% box for equation
\newcommand{\eqbox}[2]{
	
	\tcbset{colback=white,colframe=green1!100!black,title=,width=#2,arc=0pt}
	
	\begin{center}
		\begin{tcolorbox}[ams align*]
				#1
		\end{tcolorbox}
		
	\end{center} 
	
}
% END OF PREAMBLE
\usepackage{enumitem}
%%%%%%%%%%%%%%%%%%%%%%%%%%%%%%%%%%%%%

\begin{document}

\section*{Messtechnik Klausuraufgaben 2019/20}
\vspace{-2ex}
{\Large Gruppe A \vspace{2ex}}


\begin{enumerate}

  \item Nennen Sie alle Si-Basiseinheiten

  \item $a=b/c^2$, \\$b = 5\, \si{\watt} \pm 2 \%$, \\$c = 0.1 \, \si{\ampere} \pm
    0.01 \, \si{\ampere}$\\ Formel zur Berechnung des relativen Fehlers des
    Ergebnisses $a$ herleiten 

  \item Thermoelement Skizze; Gleichung zur Berechnung der Spannung (physikalisch); Beweis,
    dass die Spannung an den Übergängen zur Leitung unabhängig vom Leitermaterial ist

  \item Dimensionierung einer nichtinvertierenden Verstärkerschaltung mit einer
    Verstärkung von $V=8$; Wie lässt sich der Eingangswiderstand dieser
    Schaltung bestimmen?

  \item \begin{enumerate}[label=(\alph*)] \item Skizze Instrumentenverstärker mit 2 OPV mit möglichst maximaler
      Aussteuerung und einer Verstärkung von $V=6$ (vgl. Praktikum P04);
      \item Bestückung entweder mit LM324 oder MCP6004 (je nach Studentennummer oder so), Frage
        nach minimaler und maximaler Ausgangsspannung des Instrumentenverstärkers
        \item Einzeichnen einer Brücke mit 4 Widerständen und dessen Anschluss an die
    IV-Schaltung. Dimensionierung der Widerstände entweder zur Übersteuerung der
    OPV oder zum arbeiten im linearen Bereich
     \item Kennzeichnen  aller \emph{praktischen} Messwerte an den Ein- und Ausgängen
    der OPVs; Vergleich mit den theoretischen Messwerten und Angabe der
    Abweichung in mV
    \end{enumerate}

  \item Digitales Filter mit einer Grenzfrequenz von $f_g = 159 \, \si{\hertz}$
    soll ein am Oszilloskop gemessenes Signal filtern; Berechnung der
    Filterkoeffizienten;
    \begin{enumerate}[label=(\alph*)]
      \item Welche Parameter werden dafür benötigt?
      \item Koeffizienten berechnen
      \item Gleichung zur Anwendung des Filters angeben und deren Bestandteile erklären
    \end{enumerate}
    Skizze eines verrauschten Signals und des gefilterten Signals (in vernünftiger Skaliuerung)
  \item Praktikum P03 (Spannungsstabilisierung)

    \begin{enumerate}[label=(\alph*)]
      \item Position der Jumper;
      Einzeichnen von Messgeräten und deren Werte (aus Praktikum); Beschreiben der Schritte zur
      Ermittlung des Wirkungsgrades 
      \item Diagramme mit Verlauf des Widerstands des NTCs bei
      Abschalten/Abkühlen und Anschalten/Erwärmen, Diagramm der
      Temperaturverläufe über die Zeit, jeweils mit Messwerten beschriftet
      \item Herleitung/Methode der Bestimmung der Zeitkonstanten der Temperaturkurve (Abkühlvorgang)
      \end{enumerate}

      \begin{center}
      \vspace{-10ex}
       \includegraphics[width=\textwidth]{schaltplan} 
       \end{center}


\end{enumerate}



\end{document}
